\documentclass[12pt,UTF8,fleqn]{ctexart}
\usepackage{ctex,amsmath,amssymb,geometry,fancyhdr,bm,amsfonts,mathtools,extarrows,graphicx,url,enumerate,xcolor,float,multicol,wasysym}
\usepackage{subfigure}
\allowdisplaybreaks[4]
% 加入中文支持
\newcommand\Set[2]{\left\{#1\ \middle\vert\ #2 \right\}}
\newcommand\Lim[0]{\lim\limits_{n\rightarrow\infty}}
\newcommand\LIM[2]{\lim\limits_{#1\rightarrow#2}}
\newcommand\Ser[1]{\sum_{n=#1}^\infty}
\newcommand{\SER}[2]{\sum_{#1=#2}^\infty}
\newcommand{\Int}[4]{\varint\nolimits_{#1}^{#2}#3\mathrm d#4}
\newcommand{\aIInt}[1]{\iint\limits_{#1}}
\newcommand{\IInt}[3]{\iint\limits_{#1}#2\mathrm d#3}
\newcommand{\varIInt}[4]{\iint\limits_{#1}#2\mathrm d#3\mathrm d#4}
\newcommand{\IIInt}[3]{\iiint\limits_{#1}#2\mathrm d#3}
\newcommand{\varIIInt}[5]{\iiint\limits_{#1}#2\mathrm d#3\mathrm d#4\mathrm d#5}
\newcommand{\LInt}[3]{\varint\nolimits_{#1}#2\mathrm d#3}
\newcommand{\LOInt}[3]{\varoint\nolimits_{#1}#2\mathrm d#3}
\newcommand{\LLInt}[4]{\varint\nolimits_{#1}\nolimits^{#2}#3\mathrm d#4}
\newcommand{\BLInt}[2]{\varint\nolimits_{#1}#2}
\newcommand{\varBLInt}[3]{\varint\nolimits_{#1}\nolimits^{#2}#3}
\newcommand{\BLOInt}[2]{\varoint\nolimits_{#1}#2}
\newcommand{\SIInt}[3]{\iint\limits_{#1}#2\mathrm d#3}
\newcommand{\md}[1]{\mathrm d#1}
\newcommand{\BSIInt}[2]{\iint\limits_{#1}#2}
\newcommand{\pp}[2]{\frac{\partial #1}{\partial #2}}
\newcommand{\ppx}[1]{\frac{\partial #1}{\partial x}}
\newcommand{\ppy}[1]{\frac{\partial #1}{\partial y}}
\newcommand{\ppz}[1]{\frac{\partial #1}{\partial z}}
\newcommand{\varppx}[1]{\frac{\partial}{\partial x} #1}
\newcommand{\varppy}[1]{\frac{\partial}{\partial y} #1}
\newcommand{\varppz}[1]{\frac{\partial}{\partial z} #1}
\newcommand{\BSOIInt}[2]{\oiint\limits_{#1}#2}
\newcommand{\me}[0]{\mathrm e}
\newcommand{\m}[0]{\mathrm }
\geometry{a4paper,scale=0.80}
\pagestyle{fancy}
\rhead{常微分方程(3)}
\lhead{基础习题课期末复习}
\chead{微积分B(2)}
\begin{document}
\setcounter{section}{17}
\section{高阶线性常系数微分方程}
\subsection{复习计划}
\begin{figure}[H]
\begin{center}
\includegraphics[height=0.5\textheight]{Figures20190615/plan.png}
\end{center}
\end{figure}
\subsection{知识结构}
\begin{figure}[H]
\begin{center}
\includegraphics[height=1\textheight]{20190615-2.pdf}
\end{center}
\end{figure}
%\subsection{高阶线性微分方程解的结构}
%\begin{figure}[H]
%\begin{center}
%\includegraphics[height=0.5\textheight]{structures-1.jpg}
%\end{center}
%\end{figure}
\subsection{习题分类与解题思路}
\subsubsection{高阶线性常系数微分方程}
\begin{enumerate}
\item求解齐次方程的通解. 利用特征法,要熟记三种特征根对应的通解形式. 求解思路为:
\begin{enumerate}
\item[第一步]列出特征方程,求出特征根.
\item[第二步]根据特征根写出齐次方程的通解.

注意二阶方程和二阶以上的方程不同特征根对应的齐次方程的通解的不同形式:
\begin{enumerate}
\item二阶方程:
\begin{itemize}
\item$\lambda_1\neq\lambda_2$是两个不等实特征根$\Rightarrow$通解$y=C_1\me^{\lambda_1x}+C_2\me^{\lambda_2x}$;
\item$\lambda_1=\lambda2$是一个二重根$\Rightarrow$通解$y=C_1\me^{\lambda_1x}+C_2x\me^{\lambda_2x}$;
\item$\lambda_{1,2}=\alpha\pm\beta\m i$为一对共轭复根$\Rightarrow$通解$y=\me^{\alpha x}(C_1\cos\beta x+C_2\sin\beta x)$.
\end{itemize}
\item二阶以上的方程:
\begin{itemize}
\item$\lambda$是特征方程的单重根$\Rightarrow y=\me^{\lambda x}$是一个特解;
\item$\lambda$是特征方程的$k$重根$\Rightarrow \me^{\lambda x},x\me^{\lambda x},\cdots,x^{k-1}\me^{\lambda x}$是$k$个线性无关的特解;
\item$\lambda=\alpha\pm\beta\m i$是特征方程的一对共轭复根$\Rightarrow\me^{\alpha x}\cos\beta x,\me^{\alpha x}\sin\beta x$是两个线性无关的解;
\item$\lambda=\alpha\pm\beta\m i$是特征方程的$k$对共轭复根\\
$\Rightarrow \me^{\alpha x}\cos\alpha x,x\me^{\alpha x}\cos\alpha x,\cdots,x^{k-1}\me^{\alpha x}\cos\alpha x,\\\me^{\alpha x}\sin\beta x,x\me^{\alpha x}\sin\beta x,\cdots,x^{k-1}\me^{\alpha x}\sin\beta x$.
\end{itemize}
$n$阶齐次方程按照以上方法得到的$n$个特解线性无关.
\end{enumerate}
\end{enumerate}

【习题14.4中的1.(1)/(2)/(3)/(4)/(5)/(6).】

注意其中的1.(4),要把特征方程左侧分解成最简因式的乘积,以判断特征根的重数.
\item给定常系数线性齐次微分方程的特解,求微分方程. 关键是根据特解的形式确定特征根,从而求出特征方程,根据特征方程写出齐次微分方程.

【习题14.4中的2.(1)/(2)/(3)/(4).】
\item写出二阶线性常系数非齐次方程的特解形式. 考查求非齐次方程特解的待定系数法:
\begin{enumerate}
\item[第一步]根据自由项的形式,写出非齐次方程的特解形式:
\begin{enumerate}
\item若自由项为$f(x)=P_n\me^{\mu x}$:
\begin{itemize}
\item当$\mu$不是特征根时,可设特解为$y=Q_n(x)\me^{\mu x}$;
\item当$\mu$是单特征根时,可设特解为$y=xQ_n(x)\me^{\mu x}$;
\item当$\mu$是重特征根时,可设特解为$y=x^2Q_n(x)\me^{\mu x}$;
\end{itemize}
其中$Q_n(x)=a_nx^n+a_{n-1}x^{n-1}+\cdots+a_1x+a_0$为一般形式的$n$次多项式.
\item若自由项为$f(x)=P_n\me^{\alpha x}\cos\beta x$或$f(x)=P_n\me^{\alpha x}\sin\beta x$:
\begin{itemize}
\item当$\alpha\pm\beta\m i$不是特征根时,可设特解为$y=\me^{\alpha x}[Q_n(x)\cos\beta x+W_n(x)\sin\beta x]$;
\item当$\alpha\pm\beta\m i$是共轭特征根时,可设特解为$y=x\me^{\alpha x}[Q_n(x)\cos\beta x+W_n(x)\sin\beta x]$.
\end{itemize}
其中$Q_n(x)=a_nx^n+a_{n-1}x^{n-1}+\cdots+a_1x+a_0, W_n(x)=b_nx^n+b_{n-1}x^{n-1}+\cdots+b_1x+b_0$为一般形式的$n$次多项式.
\end{enumerate}
【习题14.4中的3.(1)/(2)/(3)/(4)/(5)/(6).】

对于含三角函数的自由项,可利用倍角公式、和差化积公式等化成待定系数法可处理的形式,如其中的3.(4)/(5).
\item[第二步]将含有待定系数的特解代入原非齐次方程,比较等号两边的系数,求出待定系数的值,从而得到非齐次方程的特解.
\end{enumerate}
\item求解二阶常系数非齐次线性微分方程的通解. 可参考以下步骤:
\begin{enumerate}
\item[第一步]写出对应齐次方程的特征方程,求出特征根,写出齐次方程的通解$y=C_1y_1(x)+C_2y_2(x)$;
\item[第二步]判断自由项$f(x)$是否为上述两种可用待定系数法处理的形式,若是,则用待定系数法求出非齐次方程的特解;

【如习题14.4中的4.(1)/(2)/(3)/(4)/(5).】
\item[第三步]若自由项$f(x)$不是上述可用待定系数法处理的形式之一,则用常数变易法求出非齐次方程的特解:
\begin{enumerate}
\item[第1步]设非齐次方程的特解为$y=C_1(x)y_1(x)+C_2(x)y_2(x)$;
\item[第2步]求解线性方程组$\begin{cases}C_1'y_1+C_2'y_2=0,\\C_1'y_1'+C_2'y_2'=f(x),\end{cases}$得到$C_1'(x),C_2'(x)$;
\item[第3步]积分$C_1'(x),C_2'(x)$得到$C_1(x)$和$C_2(x)$, 特解为$y=C_1(x)y_1(x)+C_2(x)y_2(x)$.
\end{enumerate}
注意:当已用特征法求出齐次方程的通解时,用常数变易法求特解只需代入第2步中的线性方程组求解$C_1'(x),C_2'(x)$,不需重新推导该方程.

【如习题14.3中的5.】
\item[第四步]若题目给定了初值条件,则代入初值条件,确定通解中的任意常数.

【如习题14.4中的6.】
\end{enumerate}
注意:题目中给定了初值条件时,应先求出非齐次方程的通解,用待定系数法或常数变易法求得的特解不一定满足初值条件.
\item考查欧拉方程. 可参考以下步骤(这里以二阶欧拉方程$x^2y''+axy'+by=f(x)$为例):
\begin{enumerate}
\item[第一步]当$x>0$时,令$x=\me^t$, 记$D^k=\frac{\md^k}{\md t^k}$;
\item[第二步]将欧拉方程中的$x^ky^{(k)}$用$D(D-1)\cdots(D-k+1)y$代换,自由项$f(x)$用$f(\me^t)$代换,得到线性常系数微分方程$D(D-1)y+aDy+by=f(\me^t)$, 整理成$D^2y+(a-1)Dy+by=f(\me^t)$即$y''+(a-1)y'+by=f(\me^t)$;
\item[第三步]求解该线性常系数微分方程,得到$y=C_1y_1(t)+C_2y_2(t)+y^*(t)$, 将$t$用$\ln x$代换,即可得到欧拉方程的通解$y=C_1y_1(\ln x)+C_2y_2(\ln x)+y^*(\ln x)$.
\end{enumerate}
考试时只需考虑$x>0$的情况. 当$x<0$时,令$x=-\me^t$,记$D^k=\frac{\md^k}{\md t^k}$, 与$x>0$时仅有两点区别:
\begin{itemize}
\item一是最后得到的常系数微分方程的自由项变为$f(-\me^t)$, \\
方程形式为$D^2y+(a-1)Dy+by=f(-\me^t)$;
\item二是求得该线性常系数微分方程的通解后,应将$t$用$\ln(-x)$代换, \\
通解为$y=C_1y_1[\ln(-x)]+C_2y_2[\ln(-x)]+y^*[\ln(-x)]$.
\end{itemize}

【习题14.4中的6.(1)/(2).】
\item两个应用题大家可以积累一下.

【习题14.4中的7.(1)/(2).】
\end{enumerate}
\subsubsection{一般二阶线性微分方程的常数变易法}
\begin{enumerate}
\item给定$y''+a(x)y'+b(x)y=0$的非零特解$y_1(x)$时,可直接令非齐次方程$y''+a(x)y'+b(x)y=f(x)$的特解为$y=C(x)y_1(x)$,代入非齐次方程,得到关于$C(x)$的可降阶的二阶微分方程,解出$C(x)$,即可得到非齐次方程的一个特解$y=C(x)y_1(x)$.

【如习题14.3中的4.】

这里是用齐次方程的一个特解求该齐次方程的通解,而非求非齐次方程的特解,表明常数变易法是一个很强大的方法,也可以适用于已知齐次方程的特解,求该齐次方程的通解.
\item给定$y''+a(x)y'+b(x)y=0$的通解$y=C_1y_1(x)+C_2y_2(x)$,可按照以下步骤求非齐次方程$y''+a(x)y'+b(x)y=f(x)$的特解:
\begin{enumerate}
\item[第一步]设非齐次方程的一个特解为$y=C_1(x)y_1(x)+C_2(x)y_2(x)$;
\item[第二步]求解线性方程组$\begin{cases}C_1'y_1+C_2'y_2=0\\ C_1'y_1'+C_2'y_2'=f(x),\end{cases}$得到$C_1'(x),C_2'(x)$;
\item[第三步]积分$C_1'(x),C_2'(x)$得到$C_1(x),C_2(x)$,即可求出非齐次方程的特解$y=C_1(x)y_1(x)+C_2(x)y_2(x)$.
\end{enumerate}
这个过程与常系数非齐次线性微分方程的常数变易法相同. 这里也是直接求解第二步中的线性方程组得到$C_1'(x),C_2'(x)$,不需重新推导. 可参考常系数非齐次线性微分方程的常数变易法的这个题目:

【习题14.3中的5.】
\end{enumerate}
\subsection{习题14.3解答}
\begin{enumerate}
\item[4.]已知$y_1(x)=\me^x$是齐次方程$(2x-1)y''-(2x+1)y'+2y=0$的一个解,求该方程的通解.

解:设$y=u(x)\me^x$是方程$(2x-1)y''-(2x+1)y'+2y=0$的解,

$y'=\me^x(u'+u),y''=\me^x(u''+2u'+u)$,

则$(2x-1)y''-(2x+1)y'+2y=(2x-1)\me^x(u''+2u'+u)-(2x+1)\me^x(u'+u)+2u\me^x\\
=\me^x[(2x-1)u''+(4x-2-2x-1)u']+[(2x-1)(\me^x)''-(2x+1)(\me^x)'+2\me^x]u\\
=\me^x[(2x-1)u''+(2x-3)u']=0$,

$\therefore(2x-1)u''+(2x-3)u'=0$,

令$p(x)=u'$,则$p'(x)=u''$,

$\therefore(2x-1)p'+(2x-3)p=0$(*),

当$p\not\equiv0$时$\frac{\md p}p=\frac{3-2x}{2x-1}\md x=(-1+\frac2{2x-1})\md x$,

$\therefore\ln|p|=-x+\ln|2x-1|+C$,

$\therefore p=\pm\me^C\me^{-x}(2x-1)$,

$\because p\equiv0$也满足(*)式,

$\therefore p=u'=C\me^{-x}(2x-1)$,

$\therefore u=\int C\me^{-x}(2x-1)\md x=-C\me^{-x}(2x-1)+2C\int\me^{-x}\md x=-C\me^{-x}(2x-1)-2C\me^{-x}+C_1\\
=C\me^{-x}(-2x+1-2)+C_1=C_2(2x+1)\me^{-x}+C_1$,

$\therefore$原方程的通解为$y=[C_2(2x+1)\me^{-x}+C_1]\me^x=C_1\me^x+C_2(2x+1)$.

\item[5.]已知$y_1(x)=\cos x,y_2(x)=\sin x$是齐次方程$y''+y=0$的两个解,求非齐次方程$y''+y=\sec x$的通解.

解:由1.(4)知$y_1(x),y_2(x)$线性无关,故齐次方程$y''+y=0$的通解为\\$y=C_1\cos x+C_2\sin x$,

设非齐次方程$y''+y=\sec x$的解为$y=C_1(x)\cos x+C_2(x)\sin x$,

根据常数变易法$\begin{cases}C_1'(x)\cos x+C_2'(x)\sin x=0,\\-C_1'(x)\sin x+C_2'(x)\cos x=\sec x\end{cases}$, 解得$\begin{cases}
C_1'(x)=\frac{-\tan x}{\cos^2x+\sin^2x}=-\tan x,\\
C_2'(x)=\frac1{\cos^2x+\sin^2x}=1,
\end{cases}$

$\therefore\begin{cases}
C_1(x)=-\int\tan x\md x=\ln|\cos x|+C_3,\\
C_2(x)=\int\md x=x+C_4,
\end{cases}$

$\therefore$非齐次方程$y''+y=\sec x$的通解为$y=(\ln|\cos x|)\cos x+x\sin x+C_3\cos x+C_4\sin x$.
\end{enumerate}
\subsection{习题14.4解答}
\begin{enumerate}
\item求下列齐次方程的通解:\\
\begin{tabular}{ll}
(1)$y''+6y'+9y=0$;&(2)$y''+4y'+5y=0$;\\
(2)$y^{(4)}+y'''+y'+y=0$;&(4)$3y''-2y'-8y=0$;\\
(5)$y^{(6)}+2y^{(5)}+y^{(4)}=0$;&(6)$y'''+6y''+11y'+6y=0$.
\end{tabular}

解:(1)特征方程$\lambda^2+6\lambda+9=0$的根为$\lambda_{1,2}=-3$,

$\therefore$原方程的通解为$y=C_1\me^{-3x}+C_2x\me^{-3x}$.

(2)特征方程$\lambda^2+4\lambda+5=0$的根为$\lambda_{1,2}=\frac{-4\pm\sqrt{4^2-4\times5}}2=-2\pm\mathrm i$,

$\therefore$原方程的通解为$y=\me^{-2x}(C_1\cos x+C_2\sin x)$.

(3)特征方程$\lambda^4+\lambda^3+\lambda+1=(\lambda^3+1)(\lambda+1)=(\lambda+1)^2(\lambda^2-\lambda+1)=0$的根为\\
$\lambda_{1,2}=-1,\lambda_{3,4}=\frac{1\pm\sqrt{1-4\times1\times1}}2=\frac12\pm\frac{\sqrt 3}2\m i$,

$\therefore$原方程的解为$y=C_1\me^{-x}+C_2x\me^{-x}+\me^{\frac12x}[C_3\cos(\frac{\sqrt 3}2x)+C_4\sin(\frac{\sqrt 3}2x)]$.

(4)特征方程$3\lambda^2-2\lambda-8=(3\lambda+4)(\lambda-2)=0$的根为$\lambda_1=-\frac43,\lambda_2=2$,

$\therefore$原方程的解为$y=C_1\me^{-\frac43x}+C_2\me^{2x}$.

(5)特征方程$\lambda^6+2\lambda^5+\lambda^4=\lambda^4(\lambda+1)^2=0$的根为$\lambda_{1,2,3,4}=0,\lambda_{5,6}=-1$,

$\therefore$原方程的解为$y=C_1+C_2x+C_3x^2+C_4x^3+C_5\me^{-x}+C_6x\me^{-x}$.

(6)特征方程$\lambda^3+6\lambda^2+11\lambda+6=\lambda^3+6\lambda^2+9\lambda+2\lambda+6=\lambda(\lambda+3)^2+2(\lambda+3)\\
=(\lambda+3)[\lambda(\lambda+3)+2]=(\lambda+3)(\lambda^2+3\lambda+2)=(\lambda+3)(\lambda+2)(\lambda+1)=0$,

$\therefore$特征根为$\lambda_1=-1,\lambda_2=-2,\lambda_3=-3$,

$\therefore$原方程的解为$y=C_1\me^{-x}+C_2\me^{-2x}+C_3\me^{-3x}$.

\item求出以下列函数为特解的常系数线性齐次常微分方程:\\
(1)$\me^{2x},\me^{-2x}$,二阶;\\
(2)$\me^x,x\me^x$,二阶;\\
(3)$2,\cos x,\sin  x$,三阶;\\
(4)$\me^{-x},2x\me^{-x},3\me^x$,三阶.

解:(1)特征根为$\lambda_1=2,\lambda_2=-2$,特征方程为$(\lambda-2)(\lambda+2)=\lambda^2-4=0$,

$\therefore$齐次微分方程为$y''-4y=0$.

(2)特征根为$\lambda_{1,2}=1$,特征方程为$(\lambda-1)^2=\lambda^2-2\lambda+1=0$,

$\therefore$齐次微分方程为$y''-2y+y=0$.

(3)特征根为$\lambda_1=0,\lambda_{2,3}=\pm\m i$, 特征方程为$\lambda(\lambda^2+1)=\lambda^3+\lambda=0$,

$\therefore$齐次微分方程为$y'''+y'=0$.

(4)特征根为$\lambda_{1,2}=-1,\lambda=1$,特征方程为$(\lambda+1)^2(\lambda-1)=(\lambda^2+2\lambda+1)(\lambda-1)\\
=\lambda^3+2\lambda^2+\lambda-\lambda^2-2\lambda-1=\lambda^3+\lambda^2-\lambda-1=0$,

$\therefore$齐次微分方程为$y'''+y''-y'-y=0$.

\item写出下列非齐次方程的一个特解形式:\\
(1)$y''-5y'+6y=3\me^{4x}$;\\
(2)$y''+y=(x^2-1)\me^x$;\\
(3)$y''-2y'+5y=x\me^x\cos2x$;\\
(4)$y''+4y=2\cos^22x$;\\
(5)$y''+k^2y'=k\sin(kx+2)$;\\
(6)$y^{(4)}-y'''=4$.

解:(1)该非齐次方程对应的齐次方程的特征方程$\lambda^2-5\lambda+6=(\lambda-2)(\lambda-3)=0$的根为$\lambda_1=2,\lambda_2=3$,

$\therefore$自由项$3\me^{4x}$中的$4$不是特征方程的根,

$\therefore$特解形式为$y=a\me^{4x}$.

(2)该非齐次方程对应的齐次方程的特征方程$\lambda^2+1=0$的特征根为$\lambda_{1,2}=\pm\m i$,

$\therefore$自由项$(x^2-1)\me^x$中的$1$不是特征方程的根,

$\therefore$特解形式为$y=(ax^2+bx+c)\me^x$.

(3)该非齐次方程对应的齐次方程的特征方程$\lambda^2-2\lambda+5=0$的根为$\lambda_{1,2}=\frac{2\pm\sqrt{4-4\times1\times5}}2=1\pm2\m i$,

$\therefore$自由项中的$1+2\m i$是特征方程的根,

$\therefore$特解形式为$y=x\me^x[(ax+b)\cos2x+(cx+d)\sin2x]$.

(4)$y''+4y=2\cos^22x=1+\cos4x$

方程$y''+4y=1$的自由项$1=\me^0$中的$0$不是特征方程$\lambda^2+4=0$的根,

方程$y''+4=\cos4x$的自由项$\cos4x$中的$4\m i$不是特征方程$\lambda^2+4=0$的根,

$\therefore$根据叠加原理,特解形式为$y=a+b\cos4x+c\sin4x$.

(5)自由项$k\sin(kx+2)=k\cos2\sin kx+k\sin2\cos kx$,

$\because k\m i$不是特征方程$\lambda^2+k^2\lambda=0$的根,

$\therefore$特解形式为$y=a\cos kx+b\sin kx$.

(6)自由项$4=4\me^{0x}$,故可设特解为$y=Q(x)\me^{0x}=Q(x)$,

代入原方程得$Q^{(4)}(x)-Q^{(3)}(x)=4$, 

为了使左右两侧的次数相同,该方程左侧应是与$Q^{(3)}(x)$次数相同的多项式,即$Q^{(3)}(x)$为一个零次多项式,其形式为$Q^{(3)}(x)=a, a$为待定系数,

积分三次得$Q(x)=ax^3+C_1x^2+C_2x+C_3,\ C_1,C_2,C_3$为任意常数,即$C_1,C_2,C_3$取任意常数都可满足$Q^{(4)}(x)-Q^{(3)}(x)=4$, 故不妨取$C_1=C_2=C_3=0$使得$Q(x)=ax^3$,  

因此特解形式为$y=Q(x)\me^{0x}=ax^3$.

\item求解下列二阶非齐次微分方程的解:\\
(1)$y''+2y'-3y=4x$;\\
(2)$2y''+y'-y=2\me^x$;\\
(3)$y''-3y'+2y=x\me^x$;\\
(4)$y''-3y'+2y=\cos x$;\\
(5)$y''+4y'+5y=\sin x$.

解:(1)特征方程$\lambda^2+2\lambda-3=(\lambda+3)(\lambda-1)=0$的特征根为$\lambda_1=-3,\lambda_1=1$,故齐次方程的通解为$y=C_1\me^{-3x}+C_2\me^x$,

$\because$非齐次方程的自由项为$4x=4x\me^0$, $\lambda=0$不是特征方程的解,

$\therefore$可设非齐次方程的特解为$y^*=ax+b$, 代入原非齐次方程得$0+2a-3ax-3b=4x$,

$\therefore -3a=4,2a-3b=0$,

$\therefore a=-\frac43,b=-\frac89$,

$\therefore$原非齐次方程的通解为$y=C_1\me^{-3x}+C_2\me^x-\frac43x-\frac89$.

(2)特征方程$2\lambda^2+\lambda-1=(2\lambda-1)(\lambda+1)=0$,

$\therefore$特征根$\lambda=\frac12,\lambda=-1$, 齐次方程的通解为$y=C_1\me^{\frac12x}+C_2\me^{-x}$,

$\because$原非齐次方程的自由项$\me^x$中的$\lambda=1$不是特征方程的根,

$\therefore$可设非齐次方程的特解为$y^*=a\me^x$,代入原特征方程得$2a\me^x+a\me^x-a\me^x=2\me^x$,

$\therefore a=1, y^*=\me^x$,

$\therefore$原非齐次方程的通解为$y=C_1\me^{\frac12x}+C_2\me^{-x}+\me^x$.

(3)特征方程$\lambda^2-3\lambda+2=(\lambda-1)(\lambda-2)=0$的根为$\lambda_1=1,\lambda_2=2$, 可知齐次方程的通解为$y=C_1\me^x+C_2\me^{2x}$,

$\because$原非齐次方程的自由项$x\me^x$中的$\lambda=1$是特征方程的解,

故可设非齐次方程的特解为$y^*=x(ax+b)\me^x$,

$\therefore {y^*}'=\me^x(ax^2+bx+2ax+b)=\me^x[ax^2+(2a+b)x+b],\\
{y^*}''=\me^x[ax^2+(2a+b)x+b+2ax+(2a+b)]=\me^x[ax^2+(4a+b)x+2a+2b]$, \\
代入原方程得$\me^x[ax^2+(4a+b)x+2a+2b-3ax^2-3(2a+b)x-3b+2ax^2+2bx]\\
=\me^x[(4a+b-6a-3b+2b)x+2a+2b-3b]=\me^x[-2ax+2a-b]=x\me^x$,

$\therefore -2a=1,2a-b=0$,

$\therefore a=-\frac12,b=-1$,

$\therefore {y^*}=x(-\frac12x-1)\me^x$,

$\therefore$原非齐次方程的通解为$y=C_1\me^x+C_2\me^{2x}-\frac12x(x+2)\me^x$.

(4)特征方程$\lambda^2-3\lambda+2=(\lambda-2)(\lambda-1)=0$的特征根为$\lambda_1=2,\lambda_2=1$, 故齐次方程的通解为$y=C_1\me^x+C_2\me^{2x}$,

$\because$非齐次方程的自由项$\cos x$中的$\lambda=\m i$是不是特征方程的根,

$\therefore$可设非齐次方程的特解为$y^*=a\cos x+b\sin x$,

$\therefore {y^*}'=-a\sin x+b\cos x, {y^*}''=-a\cos x-b\sin x$, 

代入原方程得$-a\cos x-b\sin x-3(-a\sin x+b\cos x)+2(a\cos x+b\sin x)=(-a-3b+2a)\cos x+(-b+3a+2b)\sin x=(a-3b)\cos x+(b+3a)\sin x=\cos x$,

$\therefore a-3b=1,b+3a=0$,

$\therefore a=\frac1{10},b=-\frac3{10},y^*=\frac1{10}(\cos x-3\sin x)$,

$\therefore$原非齐次方程的通解为$y=C_1\me^x+C_2\me^{2x}+\frac1{10}(\cos x-3\sin x)$.

(5)特征方程$\lambda^2+4\lambda+5=0$的根为$\lambda_{1,2}=\frac{-4\pm\sqrt{16-4\times5}}2=-2\pm\m i$, 故齐次方程的通解为$y=\me^{-2x}(C_1\cos x+C_2\sin x)$,

$\because$非齐次方程的自由项$\sin x$中的$\lambda=\m i$不是特征方程的根,

$\therefore$可设非齐次方程的特解为$y^*=a\cos x+b\sin x$,

${y^*}'=-a\sin x+b\cos x,{y^*}''=-a\cos x-b\sin x$,

代入原非齐次方程得$-a\cos x-b\sin x+4(-a\sin x+b\cos x)+5(a\cos x+b\sin x)=(-a+4b+5a)\cos x+(-b-4a+5b)\sin x=(4a+4b)\cos x+(-4a+4b)\sin x=\sin x$,

$\therefore 4a+4b=0,-4a+4b=1$,

$\therefore a=-\frac18,b=\frac18,y^*=-\frac18\cos x+\frac18\sin x$,

$\therefore$原非齐次方程的通解为$y=\me^{-2x}(C_1\cos x+C_2\sin x)-\frac18\cos x+\frac18\sin x$.

\item求下列二阶非齐次微分方程满足初值条件的特解:\\
(1)$y''+3y'+2y=\sin x,y(0)=y'(0)=0$;\\
(2)$y''+y'=\frac12\cos x,y(0)=y'(0)=0$.

解:(1)齐次方程$\lambda^2+3\lambda+2=(\lambda+2)(\lambda+1)=0$的根为$\lambda_1=-1,\lambda_2=-2$, 可知齐次方程的通解为$y=C_1\me^{-x}+C_2\me^{-2x}$,

该非齐次方程的自由项$\sin x$中$\lambda=\m i$不是特征方程的解, 故可设特解为$y=a\cos x+b\sin x$,

$y'=-a\sin x+b\cos x,y''=-a\cos x-b\sin x$,

代入原方程得$-a\cos x-b\sin x+3(-a\sin x+b\cos x)+2(a\cos x+b\sin x)=(-a+3b+2a)\cos x+(-b-3a+2b)\sin x=(a+3b)\cos x+(b-3a)\sin x=\sin x$,

$\therefore a+3b=0,b-3a=1$,

$\therefore a=-\frac3{10},b=\frac1{10}$, 特解为$y=-\frac3{10}\cos x+\frac1{10}\sin x$,

$\therefore$原非齐次方程的通解为$y=C_1\me^{-x}+C_2\me^{-2x}-\frac3{10}\cos x+\frac1{10}\sin x$,

$\because y(0)=C_1+C_2-\frac3{10}=0,y'(0)=-C_1-2C_2+\frac1{10}=0$,

$\therefore$满足初值条件的特解为$y(x)=\frac12\me^{-x}-\frac15\me^{-2x}-\frac3{10}\cos x+\frac1{10}\sin x$.

(2)特征方程$\lambda^2+\lambda=\lambda(\lambda+1)=0$的根为$\lambda_1=0,\lambda_2=-1$, 故齐次方程的通解为$y=C_1+C_2\me^{-x}$,

非齐次方程的自由项$\frac12\cos x$中的$\lambda=\m i$不是特征方程的根,故可设特解为$y=a\cos x+b\sin x$,

$y'=-a\sin x+b\cos x,y''=-a\cos x-b\sin x$,

代入原方程得$-a\cos x-b\sin x-a\sin x+b\cos x=(-a+b)\cos x+(-b-a)\sin x=\frac12\cos x$,

$\therefore a=-\frac14,b=\frac14$, 特解为$y=-\frac14\cos x+\frac14\sin x$,

故原方程的通解为$y=C_1+C_2\me^{-x}-\frac14\cos x+\frac14\sin x$,

$\because y(0)=C_1+C_2-\frac14,y'(0)=-C_2+\frac14=0$,

$\therefore C_1=0,C_2=\frac14$,

$\therefore$满足初值条件的特解为$\frac14\me^{-x}-\frac14\cos x+\frac14\sin x$.

\item求下列二阶微分方程的通解:\\
(1)$x^2y''+2xy'-2y=0$;\\
(2)$x^2y''+2xy'-2y=x^2+2$.

解:(1)当$x>0$时,令$x=\me^t$,则$t=\ln x$, 记$D^k=\frac{\md^k}{\md t^k}$,

则原方程可化为$D(D-1)y+2Dy-2y=0$, 即$(D^2+D-2)y=\frac{\md^2y}{\md t^2}+\frac{\md y}{\md t}-2y=0$,

特征方程$\lambda^2+\lambda-2=(\lambda+2)(\lambda-1)=0$的根为$\lambda_1=1,\lambda_2=-2$, 故$y=C_1\me^t+C_2\me^{-2t}$,

$\therefore$原方程的通解为$y=C_1x+\frac{C_2}{x^2},x>0$.

(2)当$x>0$时,令$x=\me^t$,则$t=\ln x$, 记$D^k=\frac{\md^k}{\md t^k}$, 则原方程可化为$D(D-1)y+2Dy-2y=\me^{2t}+2$,\\
即$(D^2+D-2)y=\frac{\md^2y}{\md t^2}+\frac{\md y}{\md t}-2y=\me^{2t}+2(*)$,

$\because$该方程自由项$\me^{2t}+2=\me^{2t}+2\me^{0t}$中的$\lambda=2$和$\lambda=0$都不是特征方程的根,

故可设特解为$y=a\me^{2t}+b,y'=2a\me^{2t},y''=4a\me^{2t}$,

代入方程$(*)$得$4a\me^{2t}+2a\me^{2t}-2a\me^{2t}-2b=4a\me^{2t}-2b=\me^{2t}+2$,

$\therefore 4a=1,-2b=2$,

$\therefore a=\frac14,b=-1$, 故特解为$y=\frac14\me^{2t}-1$,

$\because$方程$(*)$对应的齐次方程的通解为$y=C_1\me^t+C_2\me^{-2t}$,

$\therefore$方程$(*)$的通解为$y=C_1\me^t+C_2\me^{-2t}+\frac14\me^{2t}-1$,

$\therefore$原方程的通解为$y=C_1x+\frac{C_2}{x^2}+\frac14x^2-1,x>0$.

\item应用题:\\
(1)长度等于$6\text{m}$的链条在光滑桌面上滑动. 假定在开始时刻链条垂在桌面下的部分长度为$1\text{m}$,问链条全部滑下桌面需要多少时间?\\
(2)弹簧上端固定,下面挂有三个质量相同的重物,使弹簧伸长了$3a$. 若突然除去其中两个重物,弹簧开始自由振动,求重物的运动规律.

\begin{figure}[H]
\begin{center}
\includegraphics[height=0.5\textheight]{Figures27/14-4-7-2.pdf}
\end{center}
\caption{习题14.4 7.(1)题图示}
\end{figure}

解:(1)设$t$时刻链条垂下桌面的长度为$y=y(t)$,即$t$时刻,链条下端与桌面的垂直距离为$y(t)$,则$y(0)=1(\text{m}),y'(0)=0(\text{m/s})$,假设链条均匀,线密度为$\rho$,

则$t$时刻垂下桌面的部分的质量为$\rho y(t)$,桌面上的部分质量为$\rho[6-y(t)]$,

设垂下桌面的部分对桌面上的部分的拉力为$T$,

对于垂下桌面的部分,根据牛顿第二定律
\[\rho y(t)y''(t)=\rho y(t)\text{g}-T,\]
对于桌面上的部分,根据牛顿第二定律
\[\rho(6-y)y''(t)=T,\]
联立以上两个方程得
\[\rho y(t)y''(t)=\rho y(t)\text{g}-\rho(6-y)y'',\]
即
\[y''-\frac16\text{g}y=0,\]
该二阶齐次线性微分方程的特征方程为$\lambda^2-\frac16\text{g}=0$, 特征根为$\lambda_{1,2}=\pm\sqrt{\frac{\text{g}}6}$,

故该齐次线性微分方程的通解为$y=C_1\me^{-\sqrt{\frac{\text{g}}6}t}+C_2\me^{\sqrt{\frac{\text{g}}6}t}$,

$\because y(0)=C_1+C_2=1,y'(0)=-C_1\sqrt{\frac{\text{g}}6}+C_2\sqrt{\frac{\text{g}}6}=0$,

$\therefore C_1=C_2=\frac12$,

$\therefore y(t)=\frac12(\me^{-\sqrt{\frac{\text{g}}6}t}+\me^{\sqrt{\frac{\text{g}}6}t})=\cosh(\sqrt{\frac{\text{g}}6}t)$,

令$y(t)=6$得$\me^{\sqrt{\frac{\text{g}}6}t}=6+\sqrt{35}$或$6-\sqrt{35}$(小于1,故舍去),

$\therefore$链条全部滑下桌面用时$t=\sqrt{\frac6{\text g}}\ln(6+\sqrt{35})(\text{s})$.

\begin{figure}[H]
\begin{center}
\includegraphics[height=0.5\textheight]{Figures27/14-4-7-1.pdf}
\end{center}
\caption{习题14.4 7.(2)题图示}
\end{figure}

(2)设弹簧常数为$k$,每个重物的质量为$m$,则$3m\m g=3ak$,即$m=\frac{ak}{\m g}$,

取弹簧未挂重物时的下端为坐标原点,设$t$时刻重物离开坐标原点的位移为$y(t)$,向下为正,则$y(0)=3a,y'(0)=0$,根据牛顿第二定律,
\[my''(t)=m\m g-ky(t),\]
即\[y''+\frac kmy=\m g,\]
因$m=\frac{ak}{\m g}$,故
\begin{equation}\label{14-4-7-2}y''+\frac{\m g}ay=\m g,\end{equation}
该二阶线性非齐次微分方程的齐次方程的特征方程为$\lambda^2+\frac{\m g}a=0$,特征根为$\lambda=\pm\sqrt{\frac{\m g}a}\m i$,

故齐次方程的通解为$y=C_1\cos(\sqrt{\frac{\m g}a}t)+C_2\sin(\sqrt{\frac{\m g}a}t)$,

非齐次方程~(\ref{14-4-7-2})的自由项$\m g=\m g\me^{0t}$中的$\lambda=0$不是特征方程的根,故可设非齐次方程的特解为$y=b$, 

代入方程~(\ref{14-4-7-2})得$0+\frac{\m g}ab=\m g$, 即$b=a$,

$\therefore$非齐次方程~(\ref{14-4-7-2})的通解为$y=C_1\cos(\sqrt{\frac{\m g}a}t)+C_2\sin(\sqrt{\frac{\m g}a}t)+a$,

$\because y(0)=C_1+a=3a,y'(0)=C_2\sqrt{\frac{\m g}a}=0$,

$\therefore C_1=2a,C_2=0$,

$\therefore$重物的运动规律为$y=2a\cos(\sqrt{\frac{\m g}a}t)+a$.
\end{enumerate}
\end{document}