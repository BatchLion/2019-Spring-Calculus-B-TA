\documentclass[12pt,UTF8]{ctexart}
\usepackage{ctex,amsmath,amssymb,geometry,fancyhdr,bm,amsfonts,mathtools,extarrows,graphicx,url,enumerate,xcolor,float,multicol,wasysym}
\usepackage{subfigure}

\allowdisplaybreaks[4]
% 加入中文支持
\newcommand\Set[2]{\left\{#1\ \middle\vert\ #2 \right\}}
\newcommand\Lim[0]{\lim\limits_{n\rightarrow\infty}}
\newcommand\LIM[2]{\lim\limits_{#1\rightarrow#2}}
\newcommand\Ser[1]{\sum_{n=#1}^\infty}
\newcommand{\SER}[2]{\sum_{#1=#2}^\infty}
\newcommand{\Int}[4]{\varint\nolimits_{#1}^{#2}#3\mathrm d#4}
\newcommand{\aIInt}[1]{\iint\limits_{#1}}
\newcommand{\IInt}[3]{\iint\limits_{#1}#2\mathrm d#3}
\newcommand{\varIInt}[4]{\iint\limits_{#1}#2\mathrm d#3\mathrm d#4}
\newcommand{\IIInt}[3]{\iiint\limits_{#1}#2\mathrm d#3}
\newcommand{\varIIInt}[5]{\iiint\limits_{#1}#2\mathrm d#3\mathrm d#4\mathrm d#5}
\newcommand{\LInt}[3]{\varint\nolimits_{#1}#2\mathrm d#3}
\newcommand{\LOInt}[3]{\varoint\nolimits_{#1}#2\mathrm d#3}
\newcommand{\LLInt}[4]{\varint\nolimits_{#1}\nolimits^{#2}#3\mathrm d#4}
\newcommand{\BLInt}[2]{\varint\nolimits_{#1}#2}
\newcommand{\varBLInt}[3]{\varint\nolimits_{#1}\nolimits^{#2}#3}
\newcommand{\BLOInt}[2]{\varoint\nolimits_{#1}#2}
\newcommand{\SIInt}[3]{\iint\limits_{#1}#2\mathrm d#3}
\newcommand{\md}[1]{\mathrm d#1}
\newcommand{\BSIInt}[2]{\iint\limits_{#1}#2}
\newcommand{\pp}[2]{\frac{\partial #1}{\partial #2}}
\newcommand{\ppx}[1]{\frac{\partial #1}{\partial x}}
\newcommand{\ppy}[1]{\frac{\partial #1}{\partial y}}
\newcommand{\ppz}[1]{\frac{\partial #1}{\partial z}}
\newcommand{\varppx}[1]{\frac{\partial}{\partial x} #1}
\newcommand{\varppy}[1]{\frac{\partial}{\partial y} #1}
\newcommand{\varppz}[1]{\frac{\partial}{\partial z} #1}
\newcommand{\BSOIInt}[2]{\oiint\limits_{#1}#2}
\newcommand{\me}[0]{\mathrm e}
\geometry{a4paper,scale=0.80}
\pagestyle{fancy}
\rhead{习题13.6}
\lhead{基础习题课讲义}
\chead{微积分B(2)}
\begin{document}
\setcounter{section}{24}
\section{保守场}
\subsection{知识结构}
\noindent第13章向量场的微积分
	\begin{enumerate}
		\item[13.6]保守场
			\begin{enumerate}
				\item[13.6.1]平面保守场
				\item[13.6.2]势函数的计算
				\item[13.6.3]空间保守场
				\item[13.6.4]无源场
				\item[13.6.5]调合场
			\end{enumerate}
	\end{enumerate}
\subsection{本章小结}
这一章的标题是向量场的微积分,研究的对象是向量场。大家在今后的学习过程中会遇到很多的向量场,比如大学物理中的引力场、重力场、电场、磁场。一些数量场,比如温度场、化学反应的浓度场,可以通过求解梯度的方法转化成向量场,这些向量场也都有实用的物理意义。向量场的微积分为我们学习和研究这些向量场提供了一个数学工具。

这一章的内容可以分为四个部分,第一部分是向量场的微分,主要包括三种运算:数量场的梯度运算,得到的是一个向量场;向量场的散度运算,得到的是一个数量场;向量场的旋度运算,得到的也是一个向量场。

第二部分内容是向量场在曲线上的积分。主要是为了解决力在曲线路径上做功的问题。对于一个向量场在平面闭合路径上的积分,利用格林公式将其转化成二重积分可以简化计算。

第三部分的内容是向量场在曲面上的积分,主要是为了解决速度场的穿过曲面的通量(流量)、电场的电通量、磁场的磁通量等与通量有关的计算问题。对于向量场在空间封闭曲面上的积分,可以利用高斯公式将其转化成该向量场的散度在曲面围成区域上的三重积分,以简化计算。对于向量场在空间闭合曲线上的积分,可以利用斯托克斯公式将其转化成该向量场的旋度在曲线围成的曲面上的积分,通过适当选取曲面可以简化计算。

第四部分的内容是保守场。向量场按照是否保守可以分成保守场和非保守场两类,保守场有很多好的性质,如果可以判断一个向量场是保守场,就可以利用这些性质简化分析和计算。
\subsection{保守场}

平面保守场的性质如下所示。

\begin{figure}[H]
\begin{center}
\includegraphics[height=0.54\textheight]{Figures25/PlaneConserv.png}
\end{center}
\end{figure}

\noindent{\bf说明:}
\begin{enumerate}
\item性质\textcircled{1}中的简单路径可选为与坐标轴平行的直线或折线路径. 

\begin{figure}[H]
\begin{center}
\includegraphics[height=0.25\textheight]{Figures25/Fig13-6-Intro-2.pdf}
\end{center}
\caption{与坐标轴平行的折线路径.}
\label{13-6-Intro}
\end{figure}

在与$x$轴平行的直线路径$AC$上,积分$\varint_{L(A)}^C{\bm F\bm\cdot\md\bm l}=\varint_{L(A)}^C{X(x,y)\md x+Y(x,y)\md y}$的被积表达式$X(x,y)\md x+Y(x,y)\md y$中$\md y=0,y=y_A$,故
\[\varint\nolimits_{L(A)}\nolimits^C{\bm F\bm\cdot\md\bm l}=\varint\nolimits_{(x_A,y_A)}\nolimits^{(x_B,y_A)}{X(x,y)\md x+Y(x,y)\md y}=\Int{x_A}{x_B}{X(x,y_A)}x.\]
在与$y$轴平行的直线路径$CB$上,积分$\varint_{L(C)}^B{\bm F\bm\cdot\md\bm l}=\varint_{L(C)}^B{X(x,y)\md x+Y(x,y)\md y}$的被积表达式$X(x,y)\md x+Y(x,y)\md y$中$\md x=0,x=x_B$,故
\[\varint\nolimits_{L(C)}\nolimits^B{\bm F\bm\cdot\md\bm l}=\varint\nolimits_{(x_B,y_A)}\nolimits^{(x_B,y_B)}{X(x,y)\md x+Y(x,y)\md y}=\Int{y_A}{y_B}{Y(x_B,y)}y.\]
【考察这一点的习题:1.(1)/(2)/(3)/(4),3.(1)/(2). (第一类题目)】
%\begin{figure}[H]
%\centering 
%\subfigure[] { \label{13-6-Intro-1} 
%\includegraphics[height=0.1\textheight]{Figures25/Fig13-6-Intro-1.pdf} } 
%\subfigure[] { \label{13-6-Intro-2} 
%\includegraphics[height=0.3\textheight]{Figures25/Fig13-6-Intro-2.pdf} } 
%\caption{与坐标轴平行的直线或折线路径.} 
%\label{13-6-Intro} 
%\end{figure} 
\item\textcircled{1},\textcircled{2},\textcircled{3},\textcircled{4}相互等价,既是性质定理也是判定定理,知道其中一个即可判定该向量场是保守场,也就可以得到另外三个性质.
\item保守场是有势场,有势场也是保守场,二者是完全等价的概念. 保守场的两个性质从积分角度描述了向量场,有势场的两个性质从微分角度描述了向量场.
\item可通过对势函数这样一个数量场的分析来对向量场进行分析,比如可以用引力势能分析引力场,用重力势能分析重力场,用电势能分析电场力的场,用电势分析电场。
\item势函数与原函数相同.
\item可微的保守场是无旋场(但无旋场不一定是保守场). 

【考察这一点的习题:2. (第二类题目)】
\item平面单连通域上的无旋场是保守场. 常利用平面单连通区域上无旋来判定一个向量场是保守场.

【考察这一点的习题:1.(1)/(2)/(3)/(4),3.(1)/(2). (第一类题目)】

\item可微场才可用旋度算子计算旋度. 如一个向量场不可微,则不可用旋度算子计算旋度,难以用旋度为零来判断该向量场是保守场.

【考察这一点的习题:4. (第三类题目)】

\item求解全微分式的原函数有两种方法,变上限积分法和不定积分法.

【考察这一点的习题:3.(1)/(2). (第四类题目)】
\item空间保守场的性质与平面保守场类似,主要的不同是三维空间中的曲面单连通域(比如球壳是曲面单连通域,圆环体则不是曲面单连通域)上的无旋场是保守场。
\end{enumerate}
\subsection{无源场、无旋场}
\begin{enumerate}
\item无源场$\text{div}\bm F=\bm\nabla\bm\cdot\bm F=0$.

【与无源场有关的题目:5,7,8.(1)/(2),9. (第五类题目)】

\item无旋场$\text{rot}\bm F=\bm\nabla\times\bm F=\bm0$.

【与无旋场有关的题目:6,10.(1)/(2). (第六类题目)】

【综合题目:9. (主要考察通量的概念、高斯公式、无源场)】
\end{enumerate}
\subsection{习题13.6解答}
\begin{enumerate}
\item利用积分域与路线无关的性质计算下列积分:\\
(1)$\BLInt L{(x^3+xy^2)\md x+(y^3+x^2y)\md y}$,其中$L$为从$O(0,0)$经$A(1,1)$到$B(2,0)$的折线;\\
(2)$\BLInt L{(y+1)\tan x\md x-\ln\cos x\md y}$,其中$L$为曲线$x=\cos t,y=2\sin t(0\leqslant t\leqslant\pi)$,顺时针方向;\\
(3)$\BLInt L{(\ln\frac yx-1)\md x+\frac xy\md y}$,其中$L$为由点$A(1,1)$出发到$B(\me,3\me)$的任何一条不与$x$轴以及$y$轴相交的曲线;\\
(4)$\BLInt L{\frac{1+y^2f(xy)}y\md x+\frac x{y^2}[y^2f(xy)-1]\md y}$,其中$L$为由点$A(0,1)$出发到$B(1,2)$的任何一条不与$x$轴相交的曲线,$f$是连续可微的函数.

解:(1)\begin{figure}[H]
\begin{center}
\includegraphics[height=0.2\textheight]{Figures25/Fig13-6-1-1.pdf}
\end{center}
\caption{习题13.6 1.(1)题图示}
\label{13-6-1-1}
\end{figure}

令$\begin{cases}
X(x,y)=x^3+xy^2,\\
Y(x,y)=y^3+x^2y,
\end{cases}$则$\bm F(x,y)=(X,Y)\in C^1(\mathbb R)$, 

$\because\mathbb R$为单连通域,

且$\text{rot}\bm F=(\pp Yx-\pp Xy)\bm k=(2xy-2xy)\bm k=\bm0$,

$\therefore\BLInt L{X\md x+Y\md y}=\varint_{(0,0)}^{(2,0)}{X\md x+Y\md y}=\Int02{x^3}x=\frac14x^4\big|_0^2=4$.

(2)\begin{figure}[H]
\begin{center}
\includegraphics[height=0.3\textheight]{Figures25/Fig13-6-1-2.pdf}
\end{center}
\caption{习题13.6 1.(2)题图示}
\label{13-6-1-2}
\end{figure}

曲线$L:\begin{cases}
x=\cos t,\\
y=2\sin t,
\end{cases}(0\leqslant t\leqslant\pi)$为一椭圆弧$x^2+\frac{y^2}4=1,y\geqslant0$,顺时针的起点为$(-1,0)$终点为$(1,0)$,

设$D=(-\frac\pi2,\frac\pi2)\times(-\infty,\infty)$,则$L\in D$,

令$\begin{cases}
X(x,y)=(y+1)\tan x,\\
Y(x,y)=-\ln\cos x,
\end{cases}$则$\bm F(x,y)=(X,Y)\in C^1(D)$,

$\because D$是单连通区域,

且$\text{rot}\bm F=(\pp Yx-\pp Xy)\bm k=(-\frac{-\sin x}{\cos x}-\tan x)\bm k=\bm0$,

$\therefore\BLInt L{X\md x+Y\md y}=\varint_{(-1,0)}^{(1,0)}{X\md x+Y\md y}=\Int{-1}1{\tan x}x=0$.

(3)
\begin{figure}[H]
\begin{center}
\includegraphics[height=0.5\textheight]{Figures25/Fig13-6-1-3.pdf}
\end{center}
\caption{习题13.6 1.(3)题图示}
\label{13-6-1-3}
\end{figure}

设$D=\Set{(x,y)}{x>0,y>0}$,则$L\in D$,

令$\begin{cases}
X(x,y)=\ln\frac yx-1,\\
Y(x,y)=\frac xy,
\end{cases}$则$\bm F(x,y)=(X,Y)\in C^1(D)$,

$\because D$是单连通域,

且$\text{rot}\bm F=(\pp Yx-\pp Xy)\bm k=(\frac1y-\frac1{\frac yx}\frac1x)\bm k=\bm0$,

$\therefore\BLInt L{X\md x+Y\md y}=\varint_{(1,1)}^{(\me,3\me)}{X\md x+Y\md y}=\varint_{(1,1)}^{(\me,1)}{X\md x+Y\md y}+\varint_{(\me,1)}^{(\me,3\me)}{X\md x+Y\md y}\\
=\Int1\me{(\ln\frac1x-1)}x+\Int1{3\me}{\frac\me y}y=x(-\ln x-1)\big|_1^\me-\Int1\me x{(-\ln x-1)}+\me\ln y\big|_1^{3\me}\\
=\me(-1-1)-1\cdot(0-1)+\Int1\me{x\frac1x}x+\me\ln{3\me}-0=-2\me+1+(\me-1)+\me\ln3+\me=\me\ln3$.

(4)
\begin{figure}[H]
\begin{center}
\includegraphics[height=0.5\textheight]{Figures25/Fig13-6-1-4.pdf}
\end{center}
\caption{习题13.6 1.(4)题图示}
\label{13-6-1-4}
\end{figure}

设$D=\Set{(x,y)}{y>0}$,

令$\begin{cases}
X(x,y)=\frac{1+y^2f(xy)}y=\frac1y+yf(xy),\\
Y(x,y)=\frac x{y^2}[y^2f(xy)-1]=xf(xy)-\frac x{y^2},
\end{cases}$则$\bm F(x,y)=(X,Y)\in C^1(D)$,

$\because D$单连通,

且$\text{rot}\bm F=(\pp Yx-\pp Xy)\bm k=\{f(xy)+xf'(xy)y-\frac1{y^2}-[-\frac1{y^2}+f(xy)+yf'(xy)x]\}\bm k=\bm0$,

$\therefore\BLInt L{X\md x+Y\md y}=\varint_{(0,1)}^{(1,2)}{X\md x+Y\md y}=\varint_{(0,1)}^{(1,1)}{X\md x+Y\md y}+\varint_{(1,1)}^{(1,2)}{X\md x+Y\md y}\\
=\Int01{[1+f(x)]}x+\Int12{[f(y)-\frac1{y^2}]}y=\Int01{}x+\Int01{f(x)}x+\Int12{f(y)}y-\Int12{\frac1{y^2}}y\\
=1+\Int02{f(x)}x+\frac1y\big|_1^2=\frac12+\Int02{f(x)}x$.

\item确定$p$的值,使积分$\varint_A^B{(x^4+4xy^p)\md x+(6x^{p-1}y^2-5y^4)\md y}$与路线无关. 当$A=(0,0),B=(1,2)$时,计算积分的值.

\begin{figure}[H]
\begin{center}
\includegraphics[height=0.5\textheight]{Figures25/Fig13-6-2.pdf}
\end{center}
\caption{习题13.6 2.题图示}
\label{13-6-2}
\end{figure}

解:$\because$积分$\varint_A^B{(x^4+4xy^p)\md x+(6x^{p-1}y^2-5y^4)\md y}$与路线无关,

令$\begin{cases}
X(x,y)=x^4+4xy^p,\\
Y(x,y)=6x^{p-1}y^2-5y^4,
\end{cases}$则$\bm F(x,y)=(X,Y)$是保守场,

$\because\bm F(x,y)\in C^1$,

$\therefore\text{rot}\bm F=(\pp Yx-\pp Xy)\bm k=[6(p-1)x^{p-2}y^2-4pxy^{p-1}]\bm k=\bm0$,

$\therefore p=3$.

$\because A=(0,0),B=(1,2)$,

$\therefore\varint_A^B{X\md x+Y\md y}=\varint_{(0,0)}^{(1,2)}{X\md x+Y\md y}=\varint_{(0,0)}^{(1,0)}{X\md x+Y\md y}+\varint_{(1,0)}^{(1,2)}{X\md x+Y\md y}\\
=\Int01{x^4}x+\Int02{(6y^2-5y^4)}y=\frac15x^5\big|_0^1+(2y^3-y^5)\big|_0^2=\frac15+(2\cdot8-32)=-\frac{79}5$.

\item判定下列微分形式是否为全微分,若是,求出其原函数:\\
(1)$(2x\cos y-y^2\sin x)\md x+(2y\cos x-x^2\sin y)\md y$;\\
(2)$(\me^x\cos y+2xy^2)\md x+(2x^2y-\me^x\sin y)\md y$.

解:(1)令$\begin{cases}
X(x,y)=2x\cos y-y^2\sin x,\\
Y(x,y)=2y\cos x-x^2\sin y,
\end{cases}$则$\bm F(x,y)=(X,Y)\in C^1(\mathbb R)$,

$\because\mathbb R$是单连通区域,

且$\text{rot}\bm F=[\pp Yx-\pp Xy)\bm k=(-2y\sin x-2x\sin y-(-2x\sin y-2y\sin x)]\bm k=\bm0$,

$\therefore X\md x+Y\md y$是全微分式,

方法1:原函数$\varphi(x,y)=\varint_{(0,0)}^{(x,y)}{X(s,t)\md s+Y(s,t)\md t}+C\\
=\varint_{(0,0)}^{(x,0)}{X(s,t)\md s+Y(s,t)\md t}+\varint_{(x,0)}^{(x,y)}{X(s,t)\md s+Y(s,t)\md t}+C\\
=\Int0x{2s}s+\Int0y{(2t\cos x-x^2\sin t)}t+C=s^2\big|_0^x+(t^2\cos x+x^2\cos t)\big|_0^y\\
=x^2+(y^2\cos x-x^2\sin y-x^2)+C=y^2\cos x-x^2\sin y+C$.

方法2:设原函数为$\varphi(x,y)$,

则$\pp{\varphi(x,y)}x=2x\cos y-y^2\sin x$,

$\therefore\varphi(x,y)=x^2\cos y+y^2\cos x+C(y)$,

$\therefore\pp{\varphi(x,y)}y=-x^2\sin y+2y\cos x+C'(y)=2y\cos x-x^2\sin y$,

$\therefore C'(y)=0,\ C(y)=C$,

$\therefore\varphi(x,y)=x^2\cos y+y^2\cos x+C$.

(2)令$\begin{cases}
X(x,y)=\me^x\cos y+2xy^2,\\
Y(x,y)=2x^2y-\me^x\sin y,
\end{cases}$则$\bm F(x,y)=(X,Y)\in C^1(\mathbb R)$,

$\because\mathbb R$是单连通区域,

且$\text{rot}\bm F=(\pp Yx-\pp Xy)\bm k=[4xy-\me^x\sin y-(-\me^x\sin y+4xy)]\bm k=\bm0$,

$\therefore X\md x+Y\md y$是全微分式.

方法1:原函数$\varphi(x,y)=\varint_{(0,0)}^{(x,y)}{X(s,t)\md s+Y(s,t)\md t}+C_1\\
=\varint_{(0,0)}^{(x,0)}{X(s,t)\md s+Y(s,t)\md t}+\varint_{(x,0)}^{(x,y)}{X(s,t)\md s+Y(s,t)\md t}+C_1\\
=\Int0x{\me^s}s+\Int0y{(2x^2t-\me^x\sin t)}t+C_1=\me^s\big|_0^x+(x^2t^2+\me^x\cos t)\big|_0^y+C_1\\
=\me^x-1+(x^2y^2+\me^x\cos y-\me^x)+C_1=x^2y^2+\me^x\cos y+C$.

方法:设原函数为$\varphi(x,y)$,

则$\pp{\varphi(x,y)}x=\me^x\cos y+2xy^2$,

$\therefore\varphi(x,y)=\me^x\cos y+x^2y^2+C(y)$,

$\therefore\pp{\varphi(x,y)}y=-\me^x\sin y+2x^2y+C'(y)=2x^2y-\me^x\sin y$,

$\therefore C'(y)=0,C(y)=C$,

$\therefore\varphi(x,y)=\me^x\cos y+x^2y^2+C$.

\item设$f(u)$连续,$L$为逐段光滑简单闭曲线,求证:
\[
\BLOInt L{f(x^2+y^2)(x\md x+y\md y)}=0.
\]
证明:令$\varphi(x,y)=\frac12\Int0{x^2+y^2}{f(u)}u$,

$\pp\varphi x=f(x^2+y^2)x,\pp\varphi y=f(x^2+y^2)y$,

$\therefore\md\varphi(x,y)=\pp\varphi x\md x+\pp\varphi y\md y=f(x^2+y^2)(x\md x+y\md y)$,

$\therefore\BLOInt L{f(x^2+y^2)(x\md x+y\md y)}=0$.

5.设一元函数$f$有连续的导数,计算$\bm\nabla\bm\cdot(f(r)\bm r)$,其中
\[
\bm r=x\bm i+y\bm j+z\bm k,\ r=\sqrt{x^2+y^2+z^2},
\]
并说明$f$满足什么条件时,$f(r)\bm r$为无源场.

解:$\bm\nabla\bm\cdot(f(r)\bm r)=(\ppx{},\ppy{},\ppz{})\bm\cdot(xf(r),yf(r),zf(r))=\pp{[xf(r)]}x+\pp{[yf(r)]}y+\pp{[zf(r)]}z\\
=f(r)+xf'(r)\frac xr+f(r)+yf'(r)\frac yr+f(r)+zf'(r)\frac zr\\
=3f(r)+f'(r)\frac{x^2+y^2+z^2}r=3f(r)+rf'(r)$.

$\because f(r)\bm r$无源,

$\therefore 3f(r)+rf'(r)=0$,

当$f(r)\not\equiv0$时$\frac{\md f(r)}{f(r)}=-\frac3r\md r$,

$\therefore\ln|f(r)|=-3\ln|r|+C$,

$\therefore f(r)r^3=\pm\me^C$,

$\therefore f(r)r^3=C_0,\ C_0,C$为任意常数,$f(r)\equiv0$也满足该式.

\item设$\bm F=f(r)\bm r$($r$与$\bm r$的意义与上题同),证明$\text{rot}\bm F=\bm0$.

证明:$\text{rot}\bm F=\bm\nabla\times\bm F=\begin{vmatrix}
\bm i&\bm j&\bm k\\
\ppx{}&\ppy{}&\ppz{}\\
xf(r)&yf(r)&zf(r)
\end{vmatrix}\\
=(\ppy{zf(r)}-\ppz{yf(r)},\ppz{xf(r)}-\ppx{zf(r)},\ppx{yf(r)}-\ppy{xf(r)})\\
=(zf'(r)\frac yr-yf'(r)\frac zr,xf'(r)\frac zr-zf'(r)\frac xr,yf'(r)\frac yr-xf'(r)\frac yr)\\
=(0,0,0)=\bm0$.

\item设$f$有连续的二阶导数,计算$\bm\nabla\bm\cdot(\bm\nabla f(r))$,其中$r,\bm r$同题5,并说明$f$满足什么条件时,$\nabla f$为无源场.

解:$\bm\nabla\bm\cdot(\bm\nabla f(r))=\bm\nabla\bm\cdot(f'(r)\frac xr,f'(r)\frac yr,f'(r)\frac zr)=\varppx{[f'(r)\frac xr]}+\varppy{[f'(r)\frac yr]}+\varppz{[f'(r)\frac zr]}\\
=f''(r)\frac xr\cdot\frac xr+f'(r)\frac{r-x\frac xr}{r^2}+f''(r)\frac yr\cdot\frac yr+f'(r)\frac{r-y\frac yr}{r^2}+f''(r)\frac zr\cdot\frac zr+f'(r)\frac{r-z\frac zr}{r^2}\\
=f''(r)\frac{x^2}{r^2}+f'(r)\frac{y^2+z^2}{r^3}+f''(r)\frac{y^2}{r^2}+f'(r)\frac{z^2+x^2}{r^3}+f''(r)\frac{z^2}{r^2}+f'(r)\frac{x^2+y^2}{r^3}\\
=f''(r)\frac{x^2+y^2+z^2}{r^2}+f'(r)\frac{2(x^2+y^2+z^2)}{r^3}=f''(r)+\frac2rf'(r)$.

$\because\nabla f$为无源场,

$\therefore f''(r)+\frac2rf'(r)=0$,

当$f'(r)\not\equiv0$时$\frac{\md f'(r)}{f'(r)}=-\frac2r\md r$,

$\therefore\ln|f'(r)|=-2\ln|r|+C$,

$\therefore f'(r)r^2=\pm\me^C$,

$\therefore f'(r)=\frac{C_0}{r^2}$,$f'(r)\equiv0$也满足该式.

$\therefore f(r)=-\frac{C_0}r+C_2=\frac{C_1}r+C_2$.

\item证明下列向量场为无源场:\\
(1)$\bm v=\bm u_1\times\bm u_2$,其中$\bm u_1,\bm u_2$是无旋场;\\
(2)$\bm v=\frac{\bm r}{r^3}$,其中$r,\bm r$同题5.

证明:(1)$\text{div}\bm v=\bm\nabla\bm\cdot\bm v=\bm\nabla\bm\cdot(\bm u_1\times\bm u_2)=\bm u_2\bm\cdot(\bm\nabla\times\bm u_1)-\bm u_1\bm\cdot(\bm\nabla\times\bm u_2)\\
=\bm u_2\bm\cdot\bm0-\bm u_1\bm\cdot\bm0=0-0=0$.

{\bf注:}该公式的证明见习题13.1中的2题.

(2)$\bm\nabla\bm\cdot\bm v=\bm\nabla\bm\cdot(\frac{\bm r}{r^3})=(\ppx{},\ppy{},\ppz{})\bm\cdot(\frac x{r^3},\frac y{r^3},\frac z{r^3})=\varppx{(\frac x{r^3})}+\varppy{(\frac y{r^3})}+\varppz{(\frac z{r^3})}\\
=\frac{r^3-x3r^3\frac xr}{r^6}+\frac{r^3-y3r^3\frac yr}{r^6}+\frac{r^3-z3r^3\frac zr}{r^6}=\frac{r^2-3x^2}{r^5}+\frac{r^2-3y^2}{r^5}+\frac{r^2-3z^2}{r^5}\\
=\frac{y^2+z^2-2x^2}{r^5}+\frac{z^2+x^2-2y^2}{r^5}+\frac{x^2+y^2-2z^2}{r^5}=0$.

9.求电场$\bm v=\frac{\bm r}{r^3}$穿过包围原点的任意简单光滑闭曲面的电通量,其中$r,\bm r$同题5.

解:设$S$是包围原点的任意简单光滑闭曲面,$S_1$是$S$围成区域中的包围原点的任意简单光滑闭曲面,$S_1,S$外侧为正,记$S,S_1^-$围成的区域为$\Omega$,

则$\BSOIInt S{\bm v\bm\cdot\md\bm S}-\BSOIInt{S_1}{\bm v\bm\cdot\md\bm S}=\BSOIInt S{\bm v\bm\cdot\md\bm S}+\BSOIInt{S_1^-}{\bm v\bm\cdot\md\bm S}=\BSOIInt{S+S_1^-}{\bm v\bm\cdot\md\bm S}$,

$\because\Omega$不包含原点,

$\therefore\bm v\in C^1(\Omega)$且由上述题8(2)可知$\bm\nabla\bm\cdot\bm v=0$,

$\therefore\BSOIInt{S+S_1^-}{\bm v\bm\cdot\md\bm S}=\IIInt\Omega{\bm\nabla\bm\cdot\bm v}V=\IIInt\Omega0V=0$,

$\therefore\BSOIInt S{\bm v\bm\cdot\md\bm S}=\BSOIInt{S_1}{\bm v\bm\cdot\md\bm S}$,

$\therefore\bm v=\frac{\bm r}{r^3}$穿过包围原点的任意简单光滑闭曲面的电通量都相等,故可取一个特殊的曲面计算电通量的值.

不妨取$S_1:r=a,a>0$,记$\Omega_1$是$S_1$围成的区域,

$\therefore\BSOIInt{S_1}{\bm v\bm\cdot\md\bm S}=\BSOIInt{S_1}{\frac{\bm r}{r^3}\bm\cdot\md\bm S}=\BSOIInt{S_1}{\frac{\bm r}{a^3}\bm\cdot\md\bm S}=\frac1{a^3}\BSOIInt{S_1}{\bm r\bm\cdot\md\bm S}=\frac1{a^3}\IIInt{\Omega_1}{\bm\nabla\bm\cdot\bm r}V\\
=\frac1{a^3}\IIInt{\Omega_1}{(\pp xx+\pp yy+\pp zz)}V=\frac1{a^3}\IIInt{\Omega_1}{3}V=\frac3{a^3}\IIInt{\Omega_1}{}V=\frac3{a^3}\frac43\pi a^3=4\pi$.\footnotemark\footnotetext{该题给出了真空中点电荷电场的高斯定理的证明. 真空中位于原点的点电荷$q$产生的电场
\[\bm E=\frac q{4\pi\varepsilon_0}\frac{\bm r}{r^3}=\frac q{4\pi\varepsilon_0}\bm v,\]

故该电场穿过包围该电荷的任意简单光滑闭曲面的电通量
\[\BSOIInt S{\bm E\bm\cdot\md\bm S}=\frac q{4\pi\varepsilon_0}\BSOIInt S{\bm v\bm\cdot\md\bm S}=\frac q{\varepsilon_0}.\]

即真空中的电荷的电场穿过包围该点电荷的任意简单光滑闭曲面的电通量与该电荷的电量成正比.

}

\begin{figure}[H]
\begin{center}
\subfigure[]{\label{13-6-9-1}{\includegraphics[height=0.58\textheight]{Figures25/Fig13-6-9-1.pdf} }}
\end{center}
\end{figure}
\addtocounter{figure}{-1}
\begin{figure}[H]
\addtocounter{figure}{1}
\begin{center}
\subfigure[]{\label{13-6-9-2} {\includegraphics[height=0.6\textheight]{Figures25/Fig13-6-9-2.pdf} }}
\end{center}
\caption{习题13.6 9.题图示}
\label{13-6-9}
\end{figure}

\item证明下列向量场为无旋场:\\
(1)$\bm v=(x-x_0)\bm i+(y-y_0)\bm j+(z-z_0)\bm k$;\\
(2)$\bm v=yz(2x+y+z)\bm i+zx(x+2y+z)\bm j+xy(x+y+2z)\bm k$.

证明:(1)$\text{rot}\bm v=\bm\nabla\times\bm v=\begin{vmatrix}
\bm i&\bm j&\bm k\\
\ppx{}&\ppy{}&\ppz{}\\
x-x_0&y-y_0&z-z_0
\end{vmatrix}\\
=(\ppy{(z-z_0)}-\ppz{(y-y_0)},\ppz{(x-x_0)}-\ppx{(z-z_0)},\ppx{(y-y_0)}-\ppy{(x-x_0)})=(0,0,0)=\bm0$.

(2)$\bm v=(2xyz+y^2z+yz^2)\bm i+(zx^2+2xyz+xz^2)\bm j+(x^2y+xy^2+2xyz)\bm k$,
\[\begin{aligned}
\text{rot}\bm v&=\bm\nabla\times\bm v=\begin{vmatrix}
\bm i&\bm j&\bm k\\
\ppx{}&\ppy{}&\ppz{}\\
2xyz+y^2z+yz^2&zx^2+2xyz+xz^2&x^2y+xy^2+2xyz
\end{vmatrix}\\
&=(\ppy{(x^2y+xy^2+2xyz)}-\ppz{(zx^2+2xyz+xz^2)},\\
&\hspace{3cm}\ppz{(2xyz+y^2z+yz^2)}-\ppx{(x^2y+xy^2+2xyz)},\\
&\hspace{6cm}\ppx{(zx^2+2xyz+xz^2)}-\ppy{(2xyz+y^2z+yz^2)})\\
&=(x^2+2xy+2xz-(x^2+2xy+2zx),\\
&\hspace{3cm}2xy+y^2+2yz-(2xy+y^2+2yz),\\
&\hspace{6cm}2zx+2yz+z^2-(2zx+2yz+z^2))\\
&=\bm0.
\end{aligned}\]
\end{enumerate}
\end{document}