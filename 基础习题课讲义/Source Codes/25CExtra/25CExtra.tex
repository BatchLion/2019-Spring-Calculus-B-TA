\documentclass[12pt,UTF8]{ctexart}
\usepackage{ctex,amsmath,amssymb,geometry,fancyhdr,bm,amsfonts,mathtools,extarrows,graphicx,url,enumerate,xcolor,float,multicol,wasysym}
\usepackage{subfigure}

\allowdisplaybreaks[4]
% 加入中文支持
\newcommand\Set[2]{\left\{#1\ \middle\vert\ #2 \right\}}
\newcommand\Lim[0]{\lim\limits_{n\rightarrow\infty}}
\newcommand\LIM[2]{\lim\limits_{#1\rightarrow#2}}
\newcommand\Ser[1]{\sum_{n=#1}^\infty}
\newcommand{\SER}[2]{\sum_{#1=#2}^\infty}
\newcommand{\Int}[4]{\varint\nolimits_{#1}^{#2}#3\mathrm d#4}
\newcommand{\aIInt}[1]{\iint\limits_{#1}}
\newcommand{\IInt}[3]{\iint\limits_{#1}#2\mathrm d#3}
\newcommand{\varIInt}[4]{\iint\limits_{#1}#2\mathrm d#3\mathrm d#4}
\newcommand{\IIInt}[3]{\iiint\limits_{#1}#2\mathrm d#3}
\newcommand{\varIIInt}[5]{\iiint\limits_{#1}#2\mathrm d#3\mathrm d#4\mathrm d#5}
\newcommand{\LInt}[3]{\varint\nolimits_{#1}#2\mathrm d#3}
\newcommand{\LOInt}[3]{\varoint\nolimits_{#1}#2\mathrm d#3}
\newcommand{\LLInt}[4]{\varint\nolimits_{#1}\nolimits^{#2}#3\mathrm d#4}
\newcommand{\BLInt}[2]{\varint\nolimits_{#1}#2}
\newcommand{\varBLInt}[3]{\varint\nolimits_{#1}\nolimits^{#2}#3}
\newcommand{\BLOInt}[2]{\varoint\nolimits_{#1}#2}
\newcommand{\SIInt}[3]{\iint\limits_{#1}#2\mathrm d#3}
\newcommand{\md}[1]{\mathrm d#1}
\newcommand{\BSIInt}[2]{\iint\limits_{#1}#2}
\newcommand{\pp}[2]{\frac{\partial #1}{\partial #2}}
\newcommand{\ppx}[1]{\frac{\partial #1}{\partial x}}
\newcommand{\ppy}[1]{\frac{\partial #1}{\partial y}}
\newcommand{\ppz}[1]{\frac{\partial #1}{\partial z}}
\newcommand{\varppx}[1]{\frac{\partial}{\partial x} #1}
\newcommand{\varppy}[1]{\frac{\partial}{\partial y} #1}
\newcommand{\varppz}[1]{\frac{\partial}{\partial z} #1}
\newcommand{\BSOIInt}[2]{\oiint\limits_{#1}#2}
\newcommand{\me}[0]{\mathrm e}
\newcommand{\m}[0]{\mathrm }
\geometry{a4paper,scale=0.80}
\pagestyle{fancy}
\rhead{第13章补充题}
\lhead{基础习题课讲义}
\chead{微积分B(2)}
\begin{document}
\def\thesection{25C}
\section{第13章补充题}
\subsection{第13章补充题解答}
\begin{enumerate}
\item设$D$为$y=x,y=4x,xy=1,xy=4$所围成的区域,$F$是一元连续可微函数,$f=F'$. 求证:$\BLOInt{\partial D}{\frac{F(xy)}y}\md y=\ln2\Int14{f(v)}v$.

证明:$\because D$是第一象限内部的区域,$x$轴不穿过$D$,且$F\in C^1$,

$\therefore\frac{F(xy)}y\in C^1(D)$,

$\therefore\BLOInt{\partial D}{\frac{F(xy)}y}\md y=\varIInt D{\varppx{[\frac{F(xy)}y}-\ppy0]}xy=\varIInt D{F'(xy)}xy$,

设$\begin{cases}u=\frac yx,\\v=xy,\end{cases}$则$D=\Set{(u,v)}{1\leqslant u\leqslant 4,1\leqslant v\leqslant 4}$,

$\frac{\m D(u,v)}{\m D(x,y)}=\begin{vmatrix}-\frac y{x^2}&\frac1x\\y&x\end{vmatrix}=-\frac yx-\frac yx=-2u$,

$\therefore\varIInt D{F'(xy)}xy=\varIInt D{f(xy)}xy=\varIInt D{f(v)\frac1{|\frac{\m D(u,v)}{\m D(x,y)}|}}uv=\varIInt D{f(v)\frac1{2u}}uv=\Int14{\frac1{2u}}u\Int14{f(v)}v\\
=\frac12\ln u\big|_1^4\Int14{f(v)}v=\ln2\Int14{f(v)}v$.

\item设$D$是平面上的有界区域,函数$u(x,y)$与$v(x,y)$在$\bar D$上存在二阶连续偏导数. \\
求证:$\BLOInt{\partial D}{\begin{vmatrix}\frac{\partial u}{\partial\bm n}&\frac{\partial v}{\partial\bm n}\\ u&v\end{vmatrix}\md l}=\IInt D{\begin{vmatrix}\Delta u&\Delta v\\ u&v\end{vmatrix}}\sigma$.

证明:$\because u(x,y),v(x,y)\in C^2(D)$,

方法1:$\therefore\pp u{\bm n}=\bm n\bm\cdot\text{grad}u,\pp v{\bm n}=\bm n\bm\cdot\text{grad}v$,

$\therefore\BLOInt{\partial D}{\begin{vmatrix}\bm n\bm\cdot\text{grad}u&\bm n\bm\cdot\text{grad}v\\ u&v\end{vmatrix}\md l}=\BLOInt{\partial D}{[v(\bm n\bm\cdot\text{grad}u)-u(\bm n\bm\cdot\text{grad}v)]\md l}\\
=\BLOInt{\partial D}{(v\text{grad}u-u\text{grad}v)\bm\cdot\bm n\md l}=\BLOInt{\partial D}{(v\pp ux-u\pp vx,v\pp uy-u\pp vy)\bm\cdot\bm n\md l}\\
=\IInt D{[\varppx{(v\pp ux-u\pp vx)}+\varppy{(v\pp uy-u\pp vy)}]}\sigma\\
=\IInt D{[\pp vx\pp ux+v\frac{\partial^2u}{\partial x^2}-\pp ux\pp vx-u\frac{\partial^2v}{\partial x^2}+\pp vy\pp uy+v\frac{\partial^2u}{\partial y^2}-\pp uy\pp vy-u\frac{\partial^2v}{\partial y^2}]}\sigma\\
=\IInt D{(v\frac{\partial^2u}{\partial x^2}-u\frac{\partial^2v}{\partial x^2}+v\frac{\partial^2u}{\partial y^2}-u\frac{\partial^2v}{\partial y^2})}\sigma=\IInt D{[v(\frac{\partial^2u}{\partial x^2}+\frac{\partial^2u}{\partial y^2})-u(\frac{\partial^2v}{\partial x^2}+\frac{\partial^2v}{\partial y^2})]}\sigma\\
=\IInt D{[v\Delta u-u\Delta v]}\sigma=\IInt D{\begin{vmatrix}\Delta u&\Delta v\\ u&v\end{vmatrix}}\sigma$.

方法2:$\therefore\pp u{\bm n}=\bm\nabla u\bm\cdot\bm n,\pp v{\bm n}=\bm\nabla v\bm\cdot\bm n$,

$\therefore\BLOInt{\partial D}{\begin{vmatrix}\frac{\partial u}{\partial\bm n}&\frac{\partial v}{\partial\bm n}\\ u&v\end{vmatrix}\md l}=\BLOInt{\partial D}{(v\pp u{\bm n}-u\pp v{\bm n})\md l}=\BLOInt{\partial D}{(v\bm\nabla u\bm\cdot\bm n-u\bm\nabla v\bm\cdot\bm n)\md l}\\
=\BLOInt{\partial D}{(v\bm\nabla u-u\bm\nabla v)\bm\cdot\bm n\md l}=\IInt D{\bm\nabla\bm\cdot(v\bm\nabla u-u\bm\nabla v)}\sigma\\
=\IInt D{(\bm\nabla v\bm\cdot\bm\nabla u+v\bm\nabla\bm\cdot\bm\nabla u-\bm\nabla u\bm\cdot\bm\nabla v-u\bm\nabla\bm\cdot\bm\nabla v)}\sigma\\
=\IInt D{(v\bm\nabla\bm\cdot\bm\nabla u-u\bm\nabla\bm\cdot\bm\nabla v)}\sigma=\IInt D{(v\Delta u-u\Delta v)}\sigma=\IInt D{\begin{vmatrix}\Delta u&\Delta v\\ u&v\end{vmatrix}}\sigma$.
\item计算$I=\BSOIInt S{\frac{\cos\widehat{\bm r\bm n}}{r^2}\md S}$,其中$S$为任意光滑闭曲面,$\bm n$为$S$的外单位法向量,$M_0(x_0,y_0,z_0)$是$S$内的一个确定点,$\bm r$是连接$M_0(x_0,y_0,z_0)$和$S$上点$M(x,y,z)$的向量,$r$是$\bm r$的长度.

解:$I=\BSOIInt S{\frac{\cos\widehat{\bm r\bm n}}{r^2}\md S}=\BSOIInt S{\frac{\bm r\cdot\bm n}{r^3}\md S}$,

取$S_1:x^2+y^2+z^2=a^2$, 其中$a$足够小,使得$S_1$完全包含在$S$内,$S_1$的外侧为正,设$S$和$S_{1-}$围成的区域为$\Omega$, 设$S_1$围成的区域为$\Omega_1$,

$\because\frac{\bm r}{r^3}=\frac1{r^3}(x-x_0,y-y_0,z-z_0)\in C^1(\Omega),\ r=\sqrt{(x-x_0)^2+(y-y_0)^2+(z-z_0)^2}$,

又$\because\ppx r=\frac{2(x-x_0)}{2\sqrt{(x-x_0)^2+(y-y_0)^2+(z-z_0)^2}}=\frac{x-x_0}r,\ppy r=\frac{y-y_0}r,\ppz=\frac{z-z_0}r$,

$\therefore\BSOIInt S{\frac{\bm r\cdot\bm n}{r^3}\md S}+\BSOIInt{S_{1-}}{\frac{\bm r\cdot\bm n}{r^3}\md S}\\
=\BSOIInt{S+S_{1}-}{\frac1{r^3}(x-x_0,y-y_0,z-z_0)\cdot\bm n\md S}\\
=\varIIInt\Omega{[\varppx{(\frac{x-x_0}{r^3})}+\varppy{(\frac{y-y_0}{r^3})}+\varppz{(\frac{z-z_0}{r^3})}]}xyz\\
=\varIIInt\Omega{[\frac{r^3-(x-x_0)3r^2\frac{x-x_0}r}{r^6}+\frac{r^3-(y-y_0)3r^2\frac{y-y_0}r}{r^6}+\frac{r^3-(z-z_0)3r^2\frac{z-z_0}r}{r^6}]}xyz\\
=\varIIInt\Omega{[\frac{r^2-3(x-x_0)^2}{r^5}+\frac{r^2-3(y-y_0)^2}{r^5}+\frac{r^2-3(z-z_0)^2}{r^5}]}xyz\\
=\varIIInt\Omega{[\frac{(y-y_0)^2+(z-z_0)^2-2(x-x_0)^2}{r^5}+\frac{(z-z_0)^2+(x-x_0)^2-2(y-y_0)^2}{r^5}+\frac{(x-x_0)^2+(y-y_0)^2-2(z-z_0)^2}{r^5}]}xyz\\
=\varIIInt\Omega{0}xyz=0$,

$\therefore I=\BSOIInt S{\frac{\bm r\cdot\bm n}{r^3}\md S}=-\BSOIInt{S_{1-}}{\frac{\bm r\cdot\bm n}{r^3}\md S}=\BSOIInt{S_1}{\frac{\bm r\cdot\bm n}{r^3}\md S}=\BSOIInt{S_1}{\frac{\bm r\cdot\bm n}{a^3}\md S}=\frac1{a^3}\BSOIInt{S_1}{\bm r\cdot\bm n\md S}\\
=\frac1{a^3}\varIIInt{\Omega_1}{\bm\nabla\bm\cdot\bm r}xyz=\frac1{a^3}\varIIInt{\Omega_1}{[\varppx{(x-x_0)}+\ppy{(y-y_0)}+\ppz{(z-z_0)}]}xyz=\frac1{a^3}\varIIInt{\Omega_1}{3}xyz\\
=\frac3{a^3}\frac43\pi a^3=4\pi$.
\item设$\Omega\subset\mathbb R^3$为有界区域,其边界$\partial\Omega$为逐片光滑的闭曲面,$\bm n$是$\partial\Omega$的外单位法向量. 函数$u$和$v$在$\Omega$中有连续偏导数\footnote{这里应是二阶连续偏导数}. 求证:\\
(1)$\BSOIInt{\partial\Omega}{\pp u{\bm n}\md S}=\IIInt\Omega{\Delta u}V$;\\
(2)$\BSOIInt{\partial\Omega}{u\pp u{\bm n}\md S}=\IIInt\Omega{[(\pp ux)^2+(\pp uy)^2+(\pp uz)^2]}V+\IIInt\Omega{u\Delta u}V$;\\
(3)$\BSOIInt{\partial\Omega}{(u\pp v{\bm n}-v\pp u{\bm n})\md S}=\IIInt\Omega{(u\Delta v-v\Delta u)}V$.

证明:(1)$\because u,v\in C^2(\Omega)$,

方法1:$\therefore\BSOIInt{\partial\Omega}{\pp u{\bm n}\md S}=\BSOIInt{\partial\Omega}{\text{grad}u\bm\cdot\bm n\md S}=\BSOIInt{\partial\Omega}{(\ppx u,\ppy u,\ppz u)\bm\cdot\bm n\md S}=\IIInt\Omega{(\frac{\partial^2u}{\partial x^2}+\frac{\partial^2u}{\partial y^2}+\frac{\partial^2u}{\partial z^2})}V\\
=\IIInt\Omega{\Delta u}V$.

方法2:$\therefore\BSOIInt{\partial\Omega}{\pp u{\bm n}\md S}=\BSOIInt{\partial\Omega}{\bm\nabla u\bm\cdot\bm n\md S}=\IIInt\Omega{\bm\nabla\bm\cdot\bm\nabla u}V=\IIInt\Omega{\Delta u}V$.

(2)$\because u,v\in C^2(\Omega)$,

方法1:$\therefore\BSOIInt{\partial\Omega}{u\pp u{\bm n}\md S}=\BSOIInt{\partial\Omega}{u\text{grad}u\bm\cdot\bm n\md S}=\BSOIInt{\partial\Omega}{(u\ppx u,u\ppy u,u\ppz u)\bm\cdot\bm n\md S}\\
=\IIInt\Omega{[\varppx{(u\ppx u)}+\varppy{(u\ppy u)}+\varppz{(u\ppz u)}]}V=\IIInt\Omega{[(\ppx u)^2+u\frac{\partial^2u}{\partial x^2}+(\ppy u)^2+u\frac{\partial^2u}{\partial y^2}+(\ppz u)^2+u\frac{\partial^2u}{\partial z^2}]}V\\
=\IIInt\Omega{[(\ppx u)^2+(\ppy u)^2+(\ppz u)^2]}V+\IIInt\Omega{[u\frac{\partial^2u}{\partial x^2}+u\frac{\partial^2u}{\partial y^2}+u\frac{\partial^2u}{\partial z^2}]}V\\
=\IIInt\Omega{[(\pp ux)^2+(\pp uy)^2+(\pp uz)^2]}V+\IIInt\Omega{u\Delta u}V$.

方法2:$\therefore\BSOIInt{\partial\Omega}{u\pp u{\bm n}\md S}=\BSOIInt{\partial\Omega}{u\bm\nabla u\bm\cdot\bm n\md S}=\IIInt\Omega{\bm\nabla\bm\cdot(u\bm\nabla u)}V\\
=\IIInt\Omega{(\bm\nabla u\bm\cdot\bm\nabla u+u\bm\nabla\bm\cdot\bm\nabla u)}V=\IIInt\Omega{[(\pp ux)^2+(\pp uy)^2+(\pp uz)^2]}V+\IIInt\Omega{u\Delta u}V$.

(3)$\BSOIInt{\partial\Omega}{(u\pp v{\bm n}-v\pp u{\bm n})\md S}=\BSOIInt{\partial\Omega}{(u\text{grad}v\bm\cdot\bm n-v\text{grad}u\bm\cdot\bm n)\md S}=\BSOIInt{\partial\Omega}{(u\text{grad}v-v\text{grad}u)\bm\cdot\bm n\md S}\\
=\BSOIInt{\partial\Omega}{(u\ppx v-v\ppx u,u\ppy v-v\ppy u,u\ppz v-v\ppz u)\bm\cdot\bm n\md S}\\
=\IIInt\Omega{[\varppx{(u\ppx v-v\ppx u)}+\varppy{(u\ppy v-v\ppy u)}+\varppz{(u\ppz v-v\ppz u)}]}V\\
=\IIInt\Omega{[(\ppx u\ppx v+u\frac{\partial^2v}{\partial x^2}-\ppx v\ppx u-v\frac{\partial^2u}{\partial x^2})+(\ppy u\ppy v+u\frac{\partial^2v}{\partial y^2}-\ppy v\ppy u-v\frac{\partial^2u}{\partial y^2})\\
\hspace{3cm}+(\ppz u\ppz v+u\frac{\partial^2v}{\partial z^2}-\ppz v\ppz u-v\frac{\partial^2u}{\partial z^2})]}V\\
=\IIInt\Omega{[(u\frac{\partial^2v}{\partial x^2}-v\frac{\partial^2u}{\partial x^2})+(u\frac{\partial^2v}{\partial y^2}-v\frac{\partial^2u}{\partial y^2})+(u\frac{\partial^2v}{\partial z^2}-v\frac{\partial^2u}{\partial z^2})]}V\\
=\IIInt\Omega{[u(\frac{\partial^2v}{\partial x^2}+\frac{\partial^2v}{\partial y^2}+\frac{\partial^2v}{\partial z^2})-v(\frac{\partial^2u}{\partial x^2}+\frac{\partial^2u}{\partial y^2}+\frac{\partial^2u}{\partial z^2})]}V\\
=\IIInt\Omega{(u\Delta v-v\Delta u)}V$.

方法2:$\therefore\BSOIInt{\partial\Omega}{(u\pp v{\bm n}-v\pp u{\bm n})\md S}=\BSOIInt{\partial\Omega}{(u\bm\nabla v\bm\cdot\bm n-v\bm\nabla u\bm\cdot\bm n)\md S}=\BSOIInt{\partial\Omega}{(u\bm\nabla v-v\bm\nabla u)\bm\cdot\bm n\md S}\\
=\IIInt\Omega{\bm\nabla\bm\cdot(u\bm\nabla v-v\bm\nabla u)}V=\IIInt\Omega{(\bm\nabla u\bm\cdot\bm\nabla v+u\bm\nabla\bm\cdot\bm\nabla v-\bm\nabla v\bm\cdot\bm\nabla u-v\bm\nabla\bm\cdot\bm\nabla u)}V\\
=\IIInt\Omega{(u\bm\nabla\bm\cdot\bm\nabla v-v\bm\nabla\bm\cdot\bm\nabla u)}V=\IIInt\Omega{(u\Delta v-v\Delta u)}V$.

\item设$D$为平面区域,$u(x,y)$在$D$上有二阶连续偏导数. 求证下列命题等价:\\
(1)$u(x,y)$在$D$上是调和函数,即$\Delta u=\frac{\partial^2u}{\partial x^2}+\frac{\partial^2u}{\partial y^2}=0$;\\
(2)对于$D$内任意一条圆周$L$,如果$L$包围的区域完全属于$D$,则有$\BLOInt L{\pp u{\bm n}\md l}=0$.

证明:$(1)\Rightarrow(2)$:

$\because u(x,y)\in C^2(D)$,

$\therefore$对于$D$内任意一条圆周$L$,如果$L$包围的区域$D_L$完全属于$D$,

则$\BLOInt L{\pp u{\bm n}\md l}=\BLOInt L{\bm\nabla u\bm\cdot\bm n\md l}=\IInt{D_L}{\bm\nabla\bm\cdot\bm\nabla u}\sigma=\IInt{D_L}{\Delta u}\sigma=0$.

$(2)\Rightarrow(1)$:

方法1:$\forall(x,y)\in D$取$D$内包围$(x,y)$的圆周$L$,使得$L$包围的区域$D_L$完全属于$D$,

则$\BLOInt L{\pp u{\bm n}\md l}=0$,

$\because u(x,y)\in C^2(D)$,

$\therefore\Delta u\in C(D)$, 且$\IInt{D_L}{\Delta u}\sigma=\IInt{D_L}{\bm\nabla\bm\cdot\bm\nabla u}\sigma=\BLOInt L{\bm\nabla u\bm\cdot\bm n\md l}=\BLOInt L{\pp u{\bm n}\md l}=0$,

根据积分中值定理$\exists(\xi,\eta)\in D_L\subset D,s.t.\IInt{D_L}{\Delta u}\sigma=\Delta u(\xi,\eta)A(D_L)=0$, 其中$A(D_L)$为区域$D_L$的面积,

$\therefore\lim\limits_{D_L\rightarrow(x,y)}\frac{\IInt{D_L}{\Delta u}\sigma}{A(D_L)}=\lim\limits_{(\xi,\eta)\rightarrow(x,y)}\Delta u(\xi,\eta)=\Delta u(x,y)=0$.

方法2:反证. 若$\Delta u=\frac{\partial^2u}{\partial x^2}+\frac{\partial^2u}{\partial y^2}$在$D$上不等于零,则至少存在一点$M_0$使得
\[(\frac{\partial^2u}{\partial x^2}+\frac{\partial^2u}{\partial y^2})\Big|_{M_0}\neq0.\]
不妨设$(\frac{\partial^2u}{\partial x^2}+\frac{\partial^2u}{\partial y^2})\Big|_{M_0}>0$. 由连续性知,存在位于$D$内的、以$M_0$为中心的一个闭圆域$U$, 在$U$上处处有$(\frac{\partial^2u}{\partial x^2}+\frac{\partial^2u}{\partial y^2})>0$, 从而
\[\varIInt U{(\frac{\partial^2u}{\partial x^2}+\frac{\partial^2u}{\partial y^2})}xy>0.\]
用$\partial U$表示$U$的正向边界(这是$D$内的一个圆周),于是由格林公式得到
\[\BLOInt{\partial U}{\pp u{\bm n}\md l}=\varIInt U{(\frac{\partial^2u}{\partial x^2}+\frac{\partial^2u}{\partial y^2})}xy>0.\]

这与假设矛盾.

\item设$\Omega$为光滑曲面$S$围成的有界闭区域,$u$是$\Omega$上的调和函数$(\Delta u=\frac{\partial^2u}{\partial x^2}+\frac{\partial^2u}{\partial y^2}+\frac{\partial^2u}{\partial z^2}\equiv0)$. 求证:
\[\BSOIInt S{[\frac1r\pp u{\bm n}-u\pp{(\frac1r)}{\bm n}]\md S}=-\BSOIInt{S_\delta}{[\frac1r\pp u{\bm n}-u\pp{(\frac1r)}{\bm n}]\md S},\]
其中$M_0(x_0,y_0,z_0)$是$\Omega$内一点,$S_\delta$是以$M_0(x_0,y_0,z_0)$为中心,$\delta$为半径的球面,且$S_\delta\subset\Omega;\ \bm r=(x-x_0)\bm i+(y-y_0)\bm j+(z-z_0)\bm k, r$是$\bm r$的长度;$\bm n$是$S$的外单位法向量、$S_\delta$的内单位法向量.

证明:设$\Omega_1$为$S$内$S_\delta$外的区域,在$\Omega_1$上
\[\frac1r\pp u{\bm n}-u\pp{(\frac1r)}{\bm n}=\frac1r\bm\nabla u\bm\cdot\bm n-u\bm\nabla(\frac1r)\bm\cdot\bm n=[\frac1r\bm\nabla u-u\bm\nabla(\frac1r)]\bm\cdot\bm n,\]
$\therefore$根据高斯公式
\[\begin{aligned}
&\BSOIInt{S+S_\delta}{[\frac1r\bm\nabla u-u\bm\nabla(\frac1r)]\bm\cdot\bm n\md S}\\
=&\IIInt{\Omega_1}{\bm\nabla\bm\cdot[\frac1r\bm\nabla u-u\bm\nabla(\frac1r)]}V\\
=&\IIInt{\Omega_1}{[\bm\nabla(\frac1r)\bm\cdot\bm\nabla u+\frac1r\bm\nabla\bm\cdot\bm\nabla u-\bm\nabla u\bm\cdot\bm\nabla(\frac1r)-u\bm\nabla\bm\cdot\bm\nabla(\frac1r)]}V\\
=&\IIInt{\Omega_1}{[\frac1r\bm\nabla\bm\cdot\bm\nabla u-u\bm\nabla\bm\cdot\bm\nabla(\frac1r)]}V\\
=&\IIInt{\Omega_1}{[\frac1r\Delta u-u\Delta(\frac1r)]}V\\
=&-\IIInt{\Omega_1}{u\Delta(\frac1r)}V,
\end{aligned}\]
$\because\varppx{(\frac1r)}=\frac{-\ppx r}{r^2}=\frac1{r^2}\frac{-2(x-x_0)}{2\sqrt{(x-x_0)^2+(y-y_0)^2+(z-z_0)^2}}=\frac{-(x-x_0)}{r^3},\\
\varppy{(\frac1r)}=\frac{-(y-y_0)}{r^3},\ \varppz{(\frac1r)}=\frac{-(z-z_0)}{r^3}$,\\
$\frac{\partial^2}{\partial x^2}(\frac1r)=\varppx{[\frac{-(x-x_0)}{r^3}]}=\frac{-r^3+(x-x_0)3r^2\ppx r}{r^6}=\varppx{[\frac{-(x-x_0)}{r^3}]}=\frac{-r^3+(x-x_0)3r^2\frac{2(x-x_0)}{2\sqrt{(x-x_0)^2+(y-y_0)^2+(z-z_0)^2}}}{r^6}\\
=\varppx{[\frac{-(x-x_0)}{r^3}]}=\frac{-r^3+(x-x_0)3r^2\frac{(x-x_0)}r}{r^6}=\frac{-r^2+3(x-x_0)^2}{r^5}=\frac{2(x-x_0)^2-(y-y_0)^2-(z-z_0)^2}{r^5}$,\\
$\frac{\partial^2}{\partial y^2}(\frac1r)=\frac{2(y-y_0)^2-(z-z_0)^2-(x-x_0)^2}{r^5},\ \frac{\partial^2}{\partial z^2}(\frac1r)=\frac{2(z-z_0)^2-(x-x_0)^2-(y-y_0)^2}{r^5}$,

$\therefore\Delta(\frac1r)=\frac{\partial^2}{\partial x^2}(\frac1r)+\frac{\partial^2}{\partial y^2}(\frac1r)+\frac{\partial^2}{\partial z^2}(\frac1r)\\
=\frac{2(x-x_0)^2-(y-y_0)^2-(z-z_0)^2}{r^5}+\frac{2(y-y_0)^2-(z-z_0)^2-(x-x_0)^2}{r^5}+\frac{2(z-z_0)^2-(x-x_0)^2-(y-y_0)^2}{r^5}=0$,

$\therefore$
\[\BSOIInt{S+S_\delta}{[\frac1r\bm\nabla u-u\bm\nabla(\frac1r)]\bm\cdot\bm n\md S}=-\IIInt{\Omega_1}{u\Delta(\frac1r)}V=0,\]
$\therefore$
\[\BSOIInt{S+S_\delta}{[\frac1r\pp u{\bm n}-u\pp{(\frac1r)}{\bm n}]\bm\cdot\bm n\md S}=0,\]
$\therefore$
\[\BSOIInt S{[\frac1r\pp u{\bm n}-u\pp{(\frac1r)}{\bm n}]\md S}=-\BSOIInt{S_\delta}{[\frac1r\pp u{\bm n}-u\pp{(\frac1r)}{\bm n}]\md S}.\]
\item设$\Omega\subset\mathbb R^3$为有界区域,其边界$\partial\Omega$逐片光滑,$\bm n$是$S$的外单位法向量. $f$在$\Omega$内调和,在$\partial\Omega$上有连续的偏导数,并且在闭区域$\bar\Omega$上连续. 求证:\\
(1)$\BSOIInt{\partial\Omega}\pp f{\bm n}\md S=0$;\\
(2)$\BSOIInt{\partial\Omega}f\pp f{\bm n}\md S=\IIInt\Omega{\|\bm\nabla f\|^2}V$($\|\bm\nabla f\|$是向量$\bm\nabla f$的长度);\\
(3)若当$(x,y,z)\in\partial\Omega$时,$f(x,y,z)\equiv0$, 求证在$\Omega$内$f(x,y,z)\equiv0$.

证明:(1)$\because f$在$\Omega$内调和,即$\Delta f(x,y,z)=0,(x,y,z)\in\Omega$,

$\therefore\BSOIInt{\partial\Omega}\pp f{\bm n}\md S=\BSOIInt{\partial\Omega}{\bm\nabla f\bm\cdot\bm n\md S}=\IIInt\Omega{\bm\nabla\bm\cdot\bm\nabla f}V=\IIInt\Omega{\Delta f}V=\IIInt\Omega0V=0$.

(2)$\BSOIInt{\partial\Omega}f\pp f{\bm n}\md S=\BSOIInt{\partial\Omega}{f\bm\nabla f\bm\cdot\bm n\md S}=\IIInt\Omega{\bm\nabla\bm\cdot(f\bm\nabla f)}V=\IIInt\Omega{(\bm\nabla f\bm\cdot\bm\nabla f+f\bm\nabla\bm\cdot\bm\nabla f)}V\\
=\IIInt\Omega{(\|\bm\nabla f\|^2+f\Delta f)}V=\IIInt\Omega{\|\bm\nabla f\|^2}V$.

(3)$\because$当$(x,y,z)\in\partial\Omega$时,$f(x,y,z)\equiv0$,

$\therefore\IIInt\Omega{\|\bm\nabla f\|^2}V=\BSOIInt{\partial\Omega}f\pp f{\bm n}\md S=\BSOIInt{\partial\Omega}0\md S=0$,

$\because\|\bm\nabla f\|^2\geqslant0$,

$\therefore\|\bm\nabla f\|^2=0,\ \bm\nabla f=\bm0$,

$\therefore f(x,y,z)=Const,(x,y,z)\in\Omega$,

$\because f$在闭区域$\bar\Omega$上连续,

$\therefore f(x,y,z)=0,(x,y,z)\in\Omega$.
\item设$M_0(x_0,y_0,z_0)$为空间一确定点,$S$是点$M_0$之外的一张逐片光滑曲面. 从点$M_0$出发作射线,假定每一条这样的射线与曲面最多相交于一点,则所有与曲面$S$相交的射线构成一个锥体$\Lambda$. 以点$M_0$为中心,以任意正数$a$为半径做球,并设该球面含于锥体$\Lambda$内部的部分面积为$S_a$. 定义曲面$S$关于点$M_0$的立体角为$\Omega_S=\frac{S_a}{a^2}$.\\
(1)若$S$是以点$M_0$为中心,以任意正数$a$为半径的球面,求$\Omega_S$;\\
(2)令$\bm v=\frac{\bm r}{r^3}$, 求证:$\Omega_S=\IInt S{\bm v\bm\cdot\bm n}S$. 其中$\bm r=\overrightarrow{M_0M},r=\|\bm r\|$.

解:(1)$\Omega_S=\frac{S_a}{a^2}=\frac{4\pi a^2}{a^2}=4\pi$.

(2)证明:记半径为$a$的球面含于锥体$\Lambda$内部的部分为$S_a$, 不妨设$S_a$包含在锥体$\Lambda$的由顶点$M_0$到曲面$S$之间的部分,设$S$和$S_a$与锥体$\Lambda$侧面围成的区域为$\Omega$, 锥体$\Lambda$的侧面在$S$和$S_a$之间的部分记为$S_1$, 易知在$S_1$上$\bm v\bm\cdot\bm n=0$, 其中$\bm n$为区域$\Omega$边界上的外向单位法向量,

$\because\bm v=\frac{\bm r}{r^2}\in C^1(\Omega)$,

$\therefore\BSIInt S{\bm v\bm\cdot\bm n\md S}+\BSIInt{S_a}{\bm v\bm\cdot\bm n\md S}=\BSIInt S{\bm v\bm\cdot\bm n\md S}+\BSIInt{S_a}{\bm v\bm\cdot\bm n\md S}+\BSIInt{S_1}{\bm v\bm\cdot\bm n\md S}\\
=\BSOIInt{S+S_a+S_1}{\bm v\bm\cdot\bm n\md S}=\IIInt\Omega{\bm\nabla\bm\cdot\bm v}V=\IIInt\Omega{[\varppx{(\frac{x-x_0}{r^3})}+\varppy{(\frac{y-y_0}{r^3})}+\varppz{(\frac{z-z_0}{r^3})}]}V\\
=\IIInt\Omega{[\frac{r^3-(x-x_0)3r^2\ppx r}{r^6}+\frac{r^3-(y-y_0)3r^2\ppy r}{r^6}+\frac{r^3-(z-z_0)3r^2\ppz r}{r^6}]}V\\
=\IIInt\Omega{[\frac{r^3-(x-x_0)3r^2\frac{x-x_0}r}{r^6}+\frac{r^3-(y-y_0)3r^2\frac{y-y_0}r}{r^6}+\frac{r^3-(z-z_0)3r^2\frac{z-z_0}r}{r^6}]}V\\
=\IIInt\Omega{[\frac{r^2-3(x-x_0)^2}{r^5}+\frac{r^2-3(y-y_0)^2}{r^5}+\frac{r^2-3(z-z_0)^2}{r^5}]}V\\
=\IIInt\Omega{[\frac{(y-y_0)^2+(z-z_0)^2-2(x-x_0)^2}{r^5}+\frac{(z-z_0)^2+(x-x_0)^2-2(y-y_0)^2}{r^5}+\frac{(x-x_0)^2+(y-y_0)^2-2(z-z_0)^2}{r^5}]}V\\
=\IIInt\Omega0V=0$,

%$\text{div}(\frac{\bm r}{r^3})=\bm\nabla\bm\cdot(\frac{\bm r}{r^3})=\varppx{(\frac{x-x_0}{r^3})}+\varppy{(\frac{y-y_0}{r^3})}+\varppz{(\frac{z-z_0}{r^3})}\\
%=\frac{r^3-(x-x_0)3r^2\ppx r}{r^6}+\frac{r^3-(y-y_0)3r^2\ppy r}{r^6}+\frac{r^3-(z-z_0)3r^2\ppz r}{r^6}\\
%=\frac{r^3-(x-x_0)3r^2\frac{x-x_0}r}{r^6}+\frac{r^3-(y-y_0)3r^2\frac{y-y_0}r}{r^6}+\frac{r^3-(z-z_0)3r^2\frac{z-z_0}r}{r^6}\\
%=\frac{r^2-3(x-x_0)^2}{r^5}+\frac{r^2-3(y-y_0)^2}{r^5}+\frac{r^2-3(z-z_0)^2}{r^5}\\
%=\frac{(y-y_0)^2+(z-z_0)^2-2(x-x_0)^2}{r^5}+\frac{(z-z_0)^2+(x-x_0)^2-2(y-y_0)^2}{r^5}+\frac{(x-x_0)^2+(y-y_0)^2-2(z-z_0)^2}{r^5}\\
%=0$,

$\therefore\BSIInt S{\bm v\bm\cdot\bm n\md S}=-\BSIInt{S_a}{\bm v\bm\cdot\bm n\md S}=-\BSIInt{S_a}{\frac{\bm r}{r^3}\bm\cdot\bm n\md S}=-\BSIInt{S_a}{\frac{\bm r}{a^3}\bm\cdot\bm n\md S}=-\frac1{a^3}\BSIInt{S_a}{\bm r\bm\cdot\bm n\md S}$,

$\because$球面$S_a$上$\bm n=-(\frac xa,\frac ya,\frac za)=-\frac1a\bm r$,

$\therefore\BSIInt S{\bm v\bm\cdot\bm n\md S}=-\frac1{a^3}\BSIInt{S_a}{\bm r\bm\cdot\bm n\md S}=\frac1{a^3}\frac1a\BSIInt{S_a}{\bm r\bm\cdot\bm r\md S}=\frac1{a^3}\frac1a\BSIInt{S_a}{[(x-x_0)^2+(y-y_0)^2+(z-z_0)^2]\md S}\\
=\frac1{a^3}\frac1a\BSIInt{S_a}{a^2\md S}=\frac1{a^2}\BSIInt{S_a}{\md S}=\frac{S_a}{a^2}=\Omega_S$.
\end{enumerate}
\end{document}