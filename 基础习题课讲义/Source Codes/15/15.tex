\documentclass[12pt,UTF8]{ctexart}
\usepackage{ctex,amsmath,amssymb,geometry,fancyhdr,bm,amsfonts
,mathtools,extarrows,graphicx,url,enumerate,color,float,multicol,,subfig} 
\allowdisplaybreaks[4]
% 加入中文支持
\newcommand\Set[2]{\left\{#1\ \middle\vert\ #2 \right\}}
\newcommand\Lim[0]{\lim\limits_{n\rightarrow\infty}}
\newcommand\LIM[2]{\lim\limits_{#1\rightarrow#2}}
\newcommand\Ser[1]{\sum_{n=#1}^\infty}
\newcommand{\SER}[2]{\sum_{#1=#2}^\infty}
\newcommand{\Int}[4]{\int_{#1}^{#2}#3\mathrm d#4}
\geometry{a4paper,scale=0.80}
\pagestyle{fancy}
\rhead{习题10.4\&10.5}
\lhead{基础习题课讲义}
\chead{微积分B(2)}
\begin{document}
\setcounter{section}{14}
\section{多元函数微分学(2)}
\noindent
\subsection{知识结构}
\noindent第10章多元函数微分学
	\begin{enumerate}
		\item[10.4]复合函数微分法
		\begin{enumerate}
			\item[10.4.1]复合函数求导法则
			\item[10.4.2]函数的方向导数和梯度
				\begin{enumerate}
					\item方向导数
					\item梯度(向量)与方向导数的计算
				\end{enumerate}
			\item[10.4.3]雅可比矩阵
		\end{enumerate}
		\item[10.5]隐函数微分法
		\begin{enumerate}
			\item[10.5.1]隐函数的背景和概念
			\item[10.5.2]一个方程确定的隐函数
			\item[10.5.3]方程组确定的隐函数
		\end{enumerate}
%		\item[10.6]二元函数的泰勒公式
%		\begin{enumerate}
%			\item[10.6.1]二元函数的微分中值定理
%			\item[10.6.2]二元函数的泰勒公式
%		\end{enumerate}
	\end{enumerate}
\subsection{隐函数求导两种方法等价性的证明}
\indent设二元函数$z=z(x,y)$由方程$F(x,y,z)=0$确定,$F(x,y,z)$有连续的偏导数,求$z=z(x,y)$关于$x,y$的偏导数有以下两种方法:
\begin{enumerate}
\item[(1)]将方程$F(x,y,z)=0$两边分别对$x,y$求偏导数:
\[\frac{\partial F}{\partial x}+\frac{\partial F}{\partial z}\frac{\partial z}{\partial x}=0,\ \frac{\partial F}{\partial y}+\frac{\partial F}{\partial z}\frac{\partial z}{\partial y}=0,\]
解得
\[\frac{\partial z}{\partial x}=-\frac{F'_x}{F'_z},\ \frac{\partial z}{\partial y}=-\frac{F'_y}{F'_z}.\]
\item[(2)]将方程$F(x,y,z)=0$两边求全微分:
\[\mathrm dF(x,y,z)=F'_x\mathrm dx+F'_y\mathrm dy+F'_z\mathrm dz=0,\]
整理得
\[\mathrm dz=-\frac{F'_x}{F'_z}\mathrm dx-\frac{F'_y}{F'_z}\mathrm dy,\]
则
\[\frac{\partial z}{\partial x}=-\frac{F'_x}{F'_z},\ \frac{\partial z}{\partial y}=-\frac{F'_y}{F'_z}.\]
\end{enumerate}
\par
\indent以上这两种方法是等价的,可做如下证明。因为$z=z(x,y)$是方程$F(x,y,z)=0$确定的隐函数,所以将$z=z(x,y)$代入函数$F=F(x,y,z)$得到一个关于$x,y$的常数函数$f(x,y)=F(x,y,z(x,y))=0$(比如方程$F(x,y,z)=x^2+y^2+z^2-R^2=0$确定隐函数$z(x,y)=\sqrt{R^2-x^2-y^2}$,将$z(x,y)$代入$F(x,y,z)=x^2+y^2+z^2-R^2$得到$f(x,y)=F(x,y,z(x,y))=x^2+y^2+(\sqrt{R^2-x^2-y^2})^2-R^2\equiv0$,为一恒等于$0$的常数函数),该常数函数关于$x,y$的两个偏导数都等于$0$,即
\[\frac{\partial f}{\partial x}=\frac{\partial F}{\partial x}+\frac{\partial F}{\partial z}\frac{\partial z}{\partial z}=0,\ \frac{\partial f}{\partial y}=\frac{\partial F}{\partial y}+\frac{\partial F}{\partial z}\frac{\partial z}{\partial y}=0,\]
可据此求出$\frac{\partial z}{\partial x},\frac{\partial z}{\partial y}$,即方法(1).
\par
\indent因为$f(x,y)=F(x,y,z(x,y))=0$是一个常数函数,所以对该函关于$x,y$求全微分等于0,即
\[
\mathrm df(x,y)=(\frac{\partial F}{\partial x}+\frac{\partial F}{\partial z}\frac{\partial z}{\partial x})\mathrm dx+(\frac{\partial F}{\partial y}+\frac{\partial F}{\partial z}\frac{\partial z}{\partial y})\mathrm dy=0\mathrm dx+0\mathrm dy=0
\]
将该式做如下整理:
\begin{equation}\label{per-diff}
\begin{split}
0=\mathrm df(x,y)&=\frac{\partial F}{\partial x}\mathrm dx+\frac{\partial F}{\partial y}\mathrm dy+(\frac{\partial F}{\partial z}\frac{\partial z}{\partial x}\mathrm dx+\frac{\partial F}{\partial z}\frac{\partial z}{\partial y}\mathrm dy)\\
&=\frac{\partial F}{\partial x}\mathrm dx+\frac{\partial F}{\partial y}\mathrm dy+\frac{\partial F}{\partial z}(\frac{\partial z}{\partial x}\mathrm dx+\frac{\partial z}{\partial y}\mathrm dy)\\
&=\frac{\partial F}{\partial x}\mathrm dx+\frac{\partial F}{\partial y}\mathrm dy+\frac{\partial F}{\partial z}\mathrm dz\\
&=\mathrm dF(x,y,z)
\end{split}\end{equation}
可得到
\[
\mathrm dz=-\frac{F'_x}{F'_z}\mathrm dx-\frac{F'_y}{F'_z}\mathrm dy=\frac{\partial z}{\partial x}\mathrm dx+\frac{\partial z}{\partial y}\mathrm dy,
\]
可据此求出$\frac{\partial z}{\partial x},\frac{\partial z}{\partial y}$,即方法(2). 式~(\ref{per-diff})的推导过程即是全微分形式不变性的证明过程.
\subsection{习题10.4解答}
\begin{enumerate}
\item求下列复合函数的偏导数:\\
(1)$z=xy+xf(u),u=\frac yx$,其中$f$为$C^1$类函数,求$x\frac{\partial z}{\partial x}+y\frac{\partial z}{\partial y}$;\\
(2)$z=f(u,v),u=x,v=\frac xy$,其中$f$为$C^2$类函数,求$\frac{\partial^2z}{\partial y^2}$;\\
(3)$z=xf(\frac yx)+yg(\frac xy)$,其中$f,g$为$C^2$类函数,求$\frac{\partial^2z}{\partial x\partial y}$;\\
(4)$z=\frac y{f(x^2-y^2)}$,其中$f$为可微函数,求$\frac1x\frac{\partial z}{\partial x}+\frac1y\frac{\partial z}{\partial y}$;\\
(5)$u=f(x,xy,xyz)$,其中$f$为可微函数,求$\frac{\partial u}{\partial x},\frac{\partial u}{\partial y},\frac{\partial u}{\partial z}$;\\
(6)$z=\mathrm e^{x-2y},x=\sin t,y=t^3$,求$\frac{\mathrm dz}{\mathrm dt}$.

解:(1)$\because z=xy+xf(u)=xy+xf(\frac yx)$,

$\therefore\frac{\partial z}{\partial x}=y+f(\frac yx)+xf'(\frac yx)(-\frac y{x^2})=y+f(\frac yx)-\frac yxf'(\frac yx),\frac{\partial z}{\partial y}=x+xf'(\frac yx)\frac1x=x+f'(\frac yx)$,

$\therefore x\frac{\partial z}{\partial x}+y\frac{\partial z}{\partial y}=x[y+f(\frac yx)-\frac yxf'(\frac yx)]+y[x+f'(\frac yx)]=2xy+xf(\frac yx)$.

(2)$\frac{\partial z}{\partial y}=\frac{\partial f(u,v)}{\partial u}\frac{\partial u}{\partial y}+\frac{\partial f(u,v)}{\partial v}\frac{\partial v}{\partial y}=\frac{\partial f(u,v)}{\partial v}(-\frac x{y^2})$,

$\frac{\partial^2z}{\partial y^2}=-\frac x{y^2}[\frac{\partial^2f(u,v)}{\partial u\partial v}\frac{\partial u}{\partial y}+\frac{\partial^2f(u,v)}{\partial v^2}\frac{\partial v}{\partial y}]+\frac{\partial f(u,v)}{\partial v}\frac{2x}{y^3}=-\frac x{y^2}[\frac{\partial^2f(u,v)}{\partial v^2}(-\frac x{y^2})]+\frac{\partial f(u,v)}{\partial v}\frac{2x}{y^3}\\
=\frac{x^2}{y^4}\frac{\partial^2f(u,v)}{\partial v^2}+\frac{2x}{y^3}\frac{\partial f(u,v)}{\partial v}$.

(3)$\frac{\partial z}{\partial y}=xf'(\frac yx)\frac1x+g(\frac xy)+yg'(\frac xy)(-\frac x{y^2})=f'(\frac yx)+g(\frac xy)-\frac xyg'(\frac xy)$,

$\frac{\partial^2z}{\partial x\partial y}=f''(\frac yx)(-\frac y{x^2})+g'(\frac xy)\frac 1y-\frac1yg'(\frac xy)-\frac xyg''(\frac xy)\frac1y=-\frac y{x^2}f''(\frac yx)-\frac x{y^2}g''(\frac xy)$.

(4)$\frac{\partial z}{\partial x}=\frac{-yf'(x^2-y^2)2x}{[f(x^2-y^2)]^2}=\frac{-2xyf'(x^2-y^2)}{[f(x^2-y^2)]^2},\frac{\partial z}{\partial y}=\frac{f(x^2-y^2)-yf'(x^2-y^2)(-2y)}{[f(x^2-y^2)]^2}=\frac{f(x^2-y^2)+2y^2f'(x^2-y^2)}{[f(x^2-y^2)]^2}$,

$\frac1x\frac{\partial z}{\partial x}+\frac1y\frac{\partial z}{\partial y}=\frac1x\frac{-2xyf'(x^2-y^2)}{[f(x^2-y^2)]^2}+\frac1y\frac{f(x^2-y^2)+2y^2f'(x^2-y^2)}{[f(x^2-y^2)]^2}=\frac{f(x^2-y^2)}{y[f(x^2-y^2)]^2}=\frac1{yf(x^2-y^2)}$.

(5)$\frac{\partial u}{\partial x}=f'_1+f'_2y+f'_3yz=f'_1+yf'_2+yzf'_3,\frac{\partial u}{\partial y}=xf'_2+xzf'_3,\frac{\partial u}{\partial z}=xyf'_3$.

(6)$\frac{\mathrm dz}{\mathrm dt}=\frac{\partial z}{\partial x}\frac{\mathrm dx}{\mathrm dt}+\frac{\partial z}{\partial y}\frac{\mathrm dy}{\mathrm dt}=\mathrm e^{x-2y}\cos t+\mathrm e^{x-2y}(-2)3t^2=\mathrm e^{x-2y}(\cos t-6t^2)\\
=\mathrm e^{\sin t-2t^3}(\cos t-6t^2)$.

\item已知$z=f(x+y^2)$,其中函数$f$二阶可导,试求$\frac{\partial^2z}{\partial x^2},\frac{\partial^2z}{\partial y^2},\frac{\partial^2z}{\partial x\partial y}$.

解:$\frac{\partial z}{\partial x}=f'(x^2+y^2)2x,\frac{\partial^2z}{\partial x^2}=2f'(x^2+y^2)+2xf''(x^2+y^2)2x\\
=2f'(x^2+y^2)+4x^2f''(x^2+y^2)$,

$\frac{\partial z}{\partial y}=f'(x^2+y^2)2y,\frac{\partial^2z}{\partial y^2}=2f'(x^2+y^2)+2yf''(x^2+y^2)2y=2f'(x^2+y^2)+4y^2f''(x^2+y^2)$,

$\frac{\partial^2z}{\partial x\partial y}=2yf''(x^2+y^2)2x=4xyf''(x^2+y^2)$.

\item设$z=yf(x^2y,\frac yx)$,其中$f$具有连续的二阶偏导数,求$z''_{xx},z''_{xy}$.

解:$z'_x=y[f'_1(x^2y,\frac yx)2xy+f'_2(x^2y,\frac yx)(-\frac y{x^2})]=2xy^2f'_1(x^2y,\frac yx)-\frac{y^2}{x^2}f'_2(x^2y,\frac yx),\\
z''_{xx}=2y^2f'_1+2xy^2[f''_{11}2xy+f''_{12}(-\frac y{x^2})]-(-\frac{2y^2}{x^3})f'_2-\frac{y^2}{x^2}[f''_{21}2xy+f''_{22}(-\frac y{x^2})]\\
=2y^2f'_1+\frac{2y^2}{x^3}f'_2+4x^2y^3f''_{11}-\frac{4y^3}xf''_{12}+\frac{y^3}{x^4}f''_{22}\\
=2y^2f'_1+\frac{2y^2}{x^3}f'_2+4x^2y^3f''_{11}-\frac{4y^3}xf''_{12}+\frac{y^3}{x^4}f''_{22}$,

$z''_{xy}=4xyf'_1+2xy^2[f''_{11}x^2+f''_{12}\frac1x]-\frac{2y}{x^2}f'_2-\frac{y^2}{x^2}[f''_{21}x^2+f''_{22}\frac1x]\\
=4xyf'_1-\frac{2y}{x^2}f'_2+2x^3y^2f''_{11}+y^2f''_{12}-\frac{y^2}{x^3}f''_{22}\\
=4xyf'_1-\frac{2y}{x^2}f'_2+2x^3y^2f''_{11}+y^2f''_{12}-\frac{y^2}{x^3}f''_{22}$.

\item设函数$f,g$有连续导数,令$u=yf(\frac xy)+xg(\frac yx)$,求$x\frac{\partial^2u}{\partial x^2}+y\frac{\partial^2u}{\partial x\partial y}$.

解:{\bf【该做法应加上$f,g$有二阶连续导数的条件:】}

$\frac{\partial u}{\partial x}=yf'(\frac xy)\frac1y+g(\frac yx)+xg'(\frac yx)(-\frac y{x^2})=f'(\frac xy)+g(\frac yx)-\frac yxg'(\frac yx),\\
\frac{\partial^2u}{\partial x^2}=f''(\frac xy)\frac1y+g'(\frac yx)(-\frac y{x^2})-(-\frac y{x^2})g'(\frac yx)-\frac yxg''(\frac yx)(-\frac y{x^2})=\frac1yf''(\frac xy)+\frac{y^2}{x^3}g''(\frac yx)$,

$\frac{\partial^2u}{\partial x\partial y}=\frac{\partial^2u}{\partial y\partial x}=f''(\frac xy)(-\frac x{y^2})+g'(\frac yx)\frac1x-\frac1xg'(\frac yx)-\frac yxg''(\frac yx)\frac1x=-\frac x{y^2}f''(\frac xy)-\frac y{x^2}g''(\frac yx)$,

$\therefore x\frac{\partial^2u}{\partial x^2}+y\frac{\partial^2u}{\partial x\partial y}=x[\frac1yf''(\frac xy)+\frac{y^2}{x^3}g''(\frac yx)]+y[-\frac x{y^2}f''(\frac xy)-\frac y{x^2}g''(\frac yx)]=0$.

{\bf【正确做法:】}$\because f,g$有连续导数,

$\therefore u=yf(\frac xy)+xg(\frac yx))\in C^1(\mathbb R^2\backslash\{(x,y)|x=0\text{或}y=0\})$,

$\therefore$
\[\begin{split}
\frac{\partial u}{\partial x}&=yf'(\frac xy)\frac1y+g(\frac yx)+xg'(\frac yx)(-\frac y{x^2})=g(\frac yx)+f'(\frac xy)-\frac yxg'(\frac yx),\\
\frac{\partial u}{\partial y}&=f(\frac xy)+yf'(\frac xy)(-\frac x{y^2})+xg'(\frac yx)\frac1x=f(\frac xy)-\frac xyf'(\frac xy)+g'(\frac yx),
\end{split}\]
$\therefore$
\[
x\frac{\partial u}{\partial x}+y\frac{\partial u}{\partial y}=xg(\frac yx)+yf(\frac xy)=u\in C^1,
\]
$\therefore$
\[
\frac{\partial}{\partial x}(x\frac{\partial u}{\partial x}+y\frac{\partial u}{\partial y})-\frac{\partial u}{\partial x}=x\frac{\partial^2u}{\partial x^2}+y\frac{\partial^2u}{\partial x\partial y}=0.
\]
\item求$z=\ln(\mathrm e^{-x}+\frac{x^2}y)$在点$(1,1)$处沿$\bm v=(a,b)^T(a\neq0)$的方向导数.

解:$\because\frac{\partial z}{\partial x}=\frac{-\mathrm e^{-x}+\frac{2x}y}{\mathrm e^{-x}+\frac{x^2}y},\frac{\partial z}{\partial y}=\frac{-\frac{x^2}{y^2}}{\mathrm e^{-x}+\frac{x^2}y}$, 

$\therefore\mathrm{grad}z(1,1)=(\frac{\partial z(1,1)}{\partial x},\frac{\partial z(1,1)}{\partial y})=(\frac{-\mathrm e^{-1}+2}{\mathrm e^{-1}+1},\frac{-1}{\mathrm e^{-1}+1})=(\frac{2\mathrm e-1}{\mathrm e+1},-\frac{\mathrm e}{\mathrm e+1})$,

$\because\frac{\partial z}{\partial x},\frac{\partial z}{\partial y}$在点$(1,1)$及其附近存在且在点$(1,1)$处连续,

$\therefore f(x,y)$在点$(1,1)$处可微,

$\therefore\frac{\partial z(1,1)}{\partial\bm v}=\mathrm{grad}z(1,1)\cdot\frac{\bm v}{\|\bm v\|}=(\frac{2\mathrm e-1}{\mathrm e+1},-\frac{\mathrm e}{\mathrm e+1})\cdot\frac1{\sqrt{a^2+b^2}}(a,b)^T=\frac1{\sqrt{a^2+b^2}}(\frac{2a\mathrm e-a-b\mathrm e}{\mathrm e+1})$.

\item已知$f(x,y)=x^2-xy+y^2$.\\
(1)当$\bm v$分别为何向量时,方向导数$\frac{\partial f(1,1)}{\partial\bm v}$会取到最大值、最小值和零值?并求出其最大值和最小值.\\
(2)试求$\mathrm{grad}f(1,1)$,并说明其方向与大小的意义.

解:(1)$\because\frac{\partial f(x,y)}{\partial x}=2x-y,\frac{\partial f(x,y)}{\partial y}=2y-x$,

$\therefore\mathrm{grad}f(1,1)=(\frac{\partial f(1,1)}{\partial x},\frac{\partial f(1,1)}{\partial y})=(1,1)$,

$\because\frac{\partial z}{\partial x},\frac{\partial z}{\partial y}$在点$(1,1)$及其附近存在且在点$(1,1)$处连续,

$\therefore f(x,y)$在点$(1,1)$处可微,

$\therefore\frac{\partial f(1,1)}{\partial\bm v}=\mathrm{grad}f(1,1)\cdot\bm v=\|\mathrm{grad}f(1,1)\|\|\bm v\|\cos\theta=\|\mathrm{grad}f(1,1)\|\cos\theta$,

当$\bm v$与梯度向量的夹角$\theta=0$即$\bm v=\frac1{\sqrt2}(1,1)$时,方向导数$\frac{\partial f(1,1)}{\partial\bm v}$取得最大值$\|\mathrm{grad}f(1,1)\|=\sqrt2$;

当$\bm v$与梯度向量的夹角$\theta=\pi$即$\bm v=-\frac1{\sqrt2}(1,1)$时,方向导数$\frac{\partial f(1,1)}{\partial\bm v}$取得最小值$-\|\mathrm{grad}f(1,1)\|=-\sqrt2$;

当$\bm v$与梯度向量的夹角$\theta=\frac\pi2$即$\bm v=\frac1{\sqrt2}(-1,1)$或$\frac1{\sqrt2}(1,-1)$时,方向导数$\frac{\partial f(1,1)}{\partial\bm v}=0$.

(2)$\mathrm{grad}f(1,1)=(1,1)$,其方向表示方向导数最大的方向,其大小为方向导数的最大值.
\end{enumerate}
\subsection{习题10.5解答}
\begin{enumerate}
\item设$y=y(x),z=z(x)$是由方程$z=xf(x+y)$和$F(x,y,z)=0$所确定的函数,其中$f$和$F$分别具有连续导数和偏导数,求$\frac{\mathrm dz}{\mathrm dx}$.

解:方法1:将$z=xf(x+y),F(x,y,z)=0$两边分别对$x$求导:
\[\begin{split}
&\frac{\mathrm dz}{\mathrm dx}=f(x+y)+xf'(x+y)[1+\frac{\mathrm dy}{\mathrm dx}],\\
&F'_x+F'_y\frac{\mathrm dy}{\mathrm dx}+F'_z\frac{\mathrm dz}{\mathrm dx}=0
\end{split}\]
由该方程组解得
\[
\frac{\mathrm dz}{\mathrm dx}=\frac{[f(x+y)+xf'(x+y)]F'_y-xf'(x+y)F'_x}{F'_y+xf'(x+y)F'_z}.
\]
方法2:将$xf(x+y)-z=0,F(x,y,z)=0$两边求全微分:
\[\begin{split}
&[f(x+y)+xf'(x+y)]\mathrm dx+xf'(x+y)\mathrm dy-\mathrm dz=0,\\
&F'_x\mathrm dx+F'_y\mathrm dy+F'_z\mathrm dz=0,
\end{split}\]
因为$y=y(x),z=z(x)$,将以上方程两边分别除以$\mathrm dx$得
\[\begin{split}
&[f(x+y)+xf'(x+y)]+xf'(x+y)\frac{\mathrm dy}{\mathrm dx}-\frac{\mathrm dz}{\mathrm dx}=0,\\
&F'_x+F'_y\frac{\mathrm dy}{\mathrm dx}+F'_z\frac{\mathrm dz}{\mathrm dx}=0,
\end{split}\]
由该方程组解得
\[
\frac{\mathrm dz}{\mathrm dx}=\frac{(f+xf')F'_y-xf'F'_x}{F'_y+xf'F'_z}.
\]
\item设由方程$F(\frac xz,\frac zy)=0$可以确定隐函数$z=z(x,y)$,求$\frac{\partial z}{\partial x},\frac{\partial z}{\partial y}$.

解:方法1:将$F(\frac xz,\frac zy)=0$两边分别对$x,y$求偏导:
\[\begin{split}
F'_1(\frac1z+\frac{-x}{z^2}\frac{\partial z}{\partial x})+F'_2\frac1y\frac{\partial z}{\partial x}&=0,\\
F'_1\frac{-x}{z^2}\frac{\partial z}{\partial y}+F'_2(\frac1y\frac{\partial z}{\partial y}+\frac{-z}{y^2})&=0,\end{split}\]
$\therefore$
\[\begin{split}
\frac{\partial z}{\partial x}&=\frac{\frac1zF'_1}{\frac x{z^2}F'_1-\frac1yF'_2}=\frac{F'_1}{\frac xzF'_1-\frac zyF'_2},\\
\frac{\partial z}{\partial y}&=\frac{\frac z{y^2}F'_2}{\frac1yF'_2-\frac x{z^2}F'_1}=\frac{\frac{z^2}{y^2}F'_2}{\frac zyF'_2-\frac xzF'_1}.
\end{split}\]

方法2:将方程$F(\frac xz,\frac zy)=0$两边求全微分:
\[
F'_1\frac1z\mathrm dx+F'_2(-\frac z{y^2})\mathrm dy+[F'_1(-\frac x{z^2})+F'_2\frac1y]\mathrm dz=0,
\]
即
\[
\mathrm dz=-\frac{F'_1\frac1z}{F'_1(-\frac x{z^2})+F'_2\frac1y}\mathrm dx-\frac{F'_2(-\frac z{y^2})}{F'_1(-\frac x{z^2})+F'_2\frac1y}\mathrm dy,
\]

$\therefore$
\[\begin{split}
\frac{\partial z}{\partial x}&=\frac{-F'_1\frac1z}{F'_1(-\frac x{z^2})+F'_2\frac1y}=\frac{F'_1}{\frac xzF'_1-\frac zyF'_2},\\
\frac{\partial z}{\partial y}&=\frac{-F'_2(-\frac z{y^2})}{F'_1(-\frac x{z^2})+F'_2\frac1y}=\frac{\frac{z^2}{y^2}F'_2}{\frac zyF'_2-\frac xzF'_1}.
\end{split}\]

\item证明:方程$F(x+\frac zy,y+\frac zx)=0$所确定的隐函数$z=z(x,y)$满足方程
\[
x\frac{\partial z}{\partial x}+y\frac{\partial z}{\partial y}=z-xy.
\]

证明:方法1:

$\because$
\[
\mathrm dF(x+\frac zy,y+\frac zx)=[F'_1+F'_2(-\frac z{x^2})]\mathrm dx+[F'_1(\frac{-z}{y^2})+F'_2]\mathrm dy+(F'_1\frac1y+F'_2\frac1x)\mathrm dz=0,\]
$\therefore$
\[
\mathrm dz=-\frac{F'_1+F'_2(-\frac z{x^2})}{F'_1\frac1y+F'_2\frac1x}\mathrm dx-\frac{F'_1(\frac{-z}{y^2})+F'_2}{F'_1\frac1y+F'_2\frac1x}\mathrm dy,
\]
$\therefore$
\[
\frac{\partial z}{\partial x}=-\frac{F'_1+F'_2(-\frac z{x^2})}{F'_1\frac1y+F'_2\frac1x},\frac{\partial z}{\partial y}=-\frac{F'_1(\frac{-z}{y^2})+F'_2}{F'_1\frac1y+F'_2\frac1x},
\]
$\therefore$
\[\begin{split}
x\frac{\partial z}{\partial x}+y\frac{\partial z}{\partial y}&=-x\frac{F'_1+F'_2(-\frac z{x^2})}{F'_1\frac1y+F'_2\frac1x}-y\frac{F'_1(\frac{-z}{y^2})+F'_2}{F'_1\frac1y+F'_2\frac1x}=-\frac{xF'_1+F'_2(-\frac zx)+F'_1(\frac{-z}y)+yF'_2}{F'_1\frac1y+F'_2\frac1x}\\
&=-\frac{(x-\frac zy)F'_1+(y-\frac zx)F'_2}{F'_1\frac1y+F'_2\frac1x}=-(xy-z)\frac{\frac1yF'_1+\frac1xF'_2}{F'_1\frac1y+F'_2\frac1x}\\
&=z-xy.
\end{split}\]
方法2:将方程$F(x+\frac zy,y+\frac zx)=0$两边分别对$x,y$求偏导:
\[\begin{cases}
F'_1(1+\frac1y\frac{\partial z}{\partial x})+F'_2(\frac{-z}{x^2}+\frac1x\frac{\partial z}{\partial x})=0,\\
F'_1(\frac{-z}{y^2}+\frac1y\frac{\partial z}{\partial y})+F'_2(1+\frac1x\frac{\partial z}{\partial y})=0,
\end{cases}\]
%两式相加得
%\[
%F'_1[1-\frac z{y^2}+\frac1y(\frac{\partial z}{\partial x}+\frac{\partial z}{\partial y})]+F'_2[1-\frac z{x^2}+\frac1x(\frac{\partial z}{\partial x}+\frac{\partial z}{\partial y})]=0
%\]
这是一个关于$F'_1,F'_2$的齐次方程组,要使该方程组有非零解,则必须
\[\begin{vmatrix}
1+\frac1y\frac{\partial z}{\partial x}&\frac{-z}{x^2}+\frac1x\frac{\partial z}{\partial x}\\
\frac{-z}{y^2}+\frac1y\frac{\partial z}{\partial y}&1+\frac1x\frac{\partial z}{\partial y}
\end{vmatrix}=0,
\]
即
\[
(1+\frac1y\frac{\partial z}{\partial x})(1+\frac1x\frac{\partial z}{\partial y})-(\frac{-z}{x^2}+\frac1x\frac{\partial z}{\partial x})(\frac{-z}{y^2}+\frac1y\frac{\partial z}{\partial y})=0,
\]
整理得
\[\begin{split}
&1+\frac1y\frac{\partial z}{\partial x}+\frac1x\frac{\partial z}{\partial y}-\frac{z^2}{x^2y^2}+\frac z{x^2y}\frac{\partial z}{\partial y}+\frac z{xy^2}\frac{\partial z}{\partial x}\\
=&1-\frac{z^2}{x^2y^2}+\frac{xy+z}{xy^2}\frac{\partial z}{\partial x}+\frac{xy+z}{x^2y}\frac{\partial z}{\partial y}\\
=&\frac{(xy+z)(xy-z)}{x^2y^2}+\frac{xy+z}{xy^2}\frac{\partial z}{\partial x}+\frac{xy+z}{x^2y}\frac{\partial z}{\partial y}\\
=&\frac{xy+z}{x^2y^2}[(xy-z)+x\frac{\partial z}{\partial x}+y\frac{\partial z}{\partial y}]\\
=&0,
\end{split}\]

$\therefore xy+z=0$或$x\frac{\partial z}{\partial x}+y\frac{\partial z}{\partial y}=z-xy$,\footnotemark\footnotetext{在修订版中,这里去掉了“且$xy\neq0$”,因为由$F(x+\frac zy,y+\frac zx)=0$可直接得到$xy\neq0$.}

$\because xy+z=0$也满足$x\frac{\partial z}{\partial x}+y\frac{\partial z}{\partial y}=z-xy$,

故$z=z(x,y)$满足方程$x\frac{\partial z}{\partial x}+y\frac{\partial z}{\partial y}=z-xy$.

\item设$z=f(u)$,且$u=u(x,y)$满足$u=\varphi(u)+\int_y^xp(t)\mathrm dt$(其中$f$可导,$\varphi\in C^1$,且$\varphi'(u)\neq1,p\in C$). 求证:$p(y)\frac{\partial z}{\partial x}+p(x)\frac{\partial z}{\partial y}=0$.

证明:\[
\frac{\partial z}{\partial x}=f'(u)\frac{\partial u}{\partial x},\frac{\partial z}{\partial y}=f'(u)\frac{\partial u}{\partial y},
\]
$\because$
\[
u=\varphi(u)+\int_y^xp(t)\mathrm dt=\varphi(u)+\int_0^xp(t)\mathrm dt+\int_y^0p(t)\mathrm dt,
\]
$\therefore$
\[\frac{\partial u}{\partial x}=\varphi'(u)\frac{\partial u}{\partial x}+p(x),\frac{\partial u}{\partial y}=\varphi'(u)\frac{\partial u}{\partial y}-p(y),(*)\]
$\because\varphi'(u)\neq1$,

$\therefore$
\[
\frac{\partial u}{\partial x}=\frac{p(x)}{1-\varphi'(u)},\frac{\partial u}{\partial y}=\frac{-p(y)}{1-\varphi'(u)},
\]
$\therefore$
\[\begin{split}
\frac{\partial z}{\partial x}&=f'(u)\frac{\partial u}{\partial x}=f'(u)\frac{p(x)}{1-\varphi'(u)},\\
\frac{\partial z}{\partial y}&=f'(u)\frac{\partial u}{\partial y}=f'(u)\frac{-p(y)}{1-\varphi'(u)},
\end{split}\]
$\therefore$
\[\begin{split}
&p(y)\frac{\partial z}{\partial x}+p(x)\frac{\partial z}{\partial y}\\
=&p(y)f'(u)\frac{p(x)}{1-\varphi'(u)}+p(x)f'(u)\frac{-p(y)}{1-\varphi'(u)}\\
=&0.
\end{split}\]
\item已知方程$F(x+y,y+z)=1$确定了隐函数$z=z(x,y)$,其中$F$具有连续的二阶偏导数,求$\frac{\partial^2z}{\partial y\partial x}$.

解:方法1:将方程$F(x+y,y+z)=1$两边对$x$求偏导:
\[\begin{split}
F'_1(x+y,y+z)+F'_2(x+y,y+z)\frac{\partial z}{\partial x}=0,
\end{split}\]
得
\[
\frac{\partial z}{\partial x}=-\frac{F'_1(x+y,y+z)}{F'_2(x+y,y+z)},
\]
$\therefore$
\[
\frac{\partial^2z}{\partial y\partial x}=-\frac{[F''_{11}+F''_{12}(1+\frac{\partial z}{\partial y})]F'_2-F'_1[F''_{21}+F''_{22}(1+\frac{\partial z}{\partial y})]}{(F'_2)^2},
\]
方程$F(x+y,y+z)=1$两边对$y$求偏导:
\[\begin{split}
F'_1(x+y,y+z)+F'_2(x+y,y+z)(1+\frac{\partial z}{\partial y})=0,
\end{split}\]
得
\[
1+\frac{\partial z}{\partial y}=-\frac{F'_1(x+y,y+z)}{F'_2(x+y,y+z)},
\]
$\therefore$
\[\begin{split}
\frac{\partial^2z}{\partial y\partial x}&=-\frac{[F''_{11}+F''_{12}(-\frac{F'_1}{F'_2})]F'_2-F'_1[F''_{21}+F''_{22}(-\frac{F'_1}{F'_2})]}{(F'_2)^2}\\
&=-\frac{(F'_2)^2F''_{11}-F'_1F'_2F''_{12}-F'_1F'_2F''_{21}+(F'_1)^2F''_{22}}{(F'_2)^3}\\
&=\frac{-(F'_2)^2F''_{11}+2F'_1F'_2F''_{12}-(F'_1)^2F''_{22}}{(F'_2)^3}.
\end{split}\]
方法2:方程$F(x+y,y+z)=1$两边求全微分:
\[
\mathrm dF(x+y,y+z)=F'_1(x+y,y+z)\mathrm dx+[F'_1(x+y,y+z)+F'_2(x+y,y+z)]\mathrm dy+F'_2(x+y,y+z)\mathrm dz=0,
\]
即
\[\begin{split}
\mathrm dz&=-\frac{F'_1(x+y,y+z)}{F'_2(x+y,y+z)}\mathrm dx-\frac{F'_1(x+y,y+z)+F'_2(x+y,y+z)}{F'_2(x+y,y+z)}\mathrm dz\\
&=-\frac{F'_1(x+y,y+z)}{F'_2(x+y,y+z)}\mathrm dx-[1+\frac{F'_1(x+y,y+z)}{F'_2(x+y,y+z)}]\mathrm dz,
\end{split}\]
$\therefore$
\[
\frac{\partial z}{\partial x}=-\frac{F'_1(x+y,y+z)}{F'_2(x+y,y+z)},
\]
$\therefore$
\[
\frac{\partial^2z}{\partial y\partial x}=-\frac{[F''_{11}+F''_{12}(1+\frac{\partial z}{\partial y})]F'_2-F'_1[F''_{21}+F''_{22}(1+\frac{\partial z}{\partial y})]}{(F'_2)^2},
\]
$\because$
\[
\frac{\partial z}{\partial y}=-1-\frac{F'_1(x+y,y+z)}{F'_2(x+y,y+z)},
\]
$\therefore$
\[\begin{split}
\frac{\partial^2z}{\partial y\partial x}&=-\frac{[F''_{11}+F''_{12}(-\frac{F'_1}{F'_2})]F'_2-F'_1[F''_{21}+F''_{22}(-\frac{F'_1}{F'_2})]}{(F'_2)^2}\\
&=-\frac{(F'_2)^2F''_{11}-F'_1F'_2F''_{12}-F'_1F'_2F''_{21}+(F'_1)^2F''_{22}}{(F'_2)^3}\\
&=\frac{-(F'_2)^2F''_{11}+2F'_1F'_2F''_{12}-(F'_1)^2F''_{22}}{(F'_2)^3}.
\end{split}\]
\item设方程组$\begin{cases}
x^2+y^2+z^2=3x,\\
2x-3y+5z=4,
\end{cases}$确定$y$与$z$是$x$的函数,求$\frac{\mathrm dy}{\mathrm dx},\frac{\mathrm dz}{\mathrm dx}$.

解:方法1:将方程组的两个方程两边分别对$x$求导:

$\begin{cases}
2x+2y\frac{\mathrm dy}{\mathrm dx}+2z\frac{\mathrm dz}{\mathrm dx}=3,\\
2-3\frac{\mathrm dy}{\mathrm dx}+5\frac{\mathrm dz}{\mathrm dx}=0,
\end{cases}$

可解得
$\frac{\mathrm dy}{\mathrm dx}=\frac{\begin{vmatrix}3-2x&2z\\-2&5\end{vmatrix}}{\begin{vmatrix}2y&2z\\-3&5\end{vmatrix}}=\frac{15-10x+4z}{10y+6z},\frac{\mathrm dz}{\mathrm dy}=\frac{\begin{vmatrix}2y&3-2x\\-3&-2\end{vmatrix}}{\begin{vmatrix}2y&2z\\-3&5\end{vmatrix}}=\frac{-4y+9-6x}{10y+6z}$.

方法2:将方程组的两个方程两边分别求全微分:

$\begin{cases}
2\mathrm dx+2y\mathrm dy+2z\mathrm dz=3,\\
2\mathrm dx-3\mathrm dy+5\mathrm dz=0,
\end{cases}$

$\because y$与$z$是$x$的函数

$\therefore\begin{cases}
2x+2y\frac{\mathrm dy}{\mathrm dx}+2z\frac{\mathrm dz}{\mathrm dx}=3,\\
2-3\frac{\mathrm dy}{\mathrm dx}+5\frac{\mathrm dz}{\mathrm dx}=0,
\end{cases}$

可解得
$\frac{\mathrm dy}{\mathrm dx}=\frac{\begin{vmatrix}3-2x&2z\\-2&5\end{vmatrix}}{\begin{vmatrix}2y&2z\\-3&5\end{vmatrix}}=\frac{15-10x+4z}{10y+6z},\frac{\mathrm dz}{\mathrm dy}=\frac{\begin{vmatrix}2y&3-2x\\-3&-2\end{vmatrix}}{\begin{vmatrix}2y&2z\\-3&5\end{vmatrix}}=\frac{-4y+9-6x}{10y+6z}$.
\end{enumerate}
\end{document}